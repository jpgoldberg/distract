%%
%% General documentation for distrat/askmap
%%
\def\version{0.74d}
\def\draftdate{February 3, 1996}
\documentclass[%
	11pt,
        a4paper,
%	letterpaper,
        twoside]{workrep}
\usepackage{array,verbatim,xspace,alltt,multicol,ifthen}
\usepackage{moreverb,path,lastpage}
\usepackage{float}
\floatstyle{ruled}
\restylefloat{figure}

%%
%% Set up running headers (should really go into workrep.cls)
\usepackage{fancyheadings}
\pagestyle{fancyplain}
\renewcommand{\chaptermark}[1]{%
  \markboth{\chaptername\ \thechapter.\ #1}{}}  % Chapter N. Title
\renewcommand{\sectionmark}[1]{%
   \markright{\textit{\thesection}\ #1}{}}  % \textit{N.N} Title

\lhead[\fancyplain{}{\textbf{\thepage}}]
      {\fancyplain{}{\textbf{\rightmark}}}
\chead{\fancyplain{}{}}
\rhead[\fancyplain{}{\textbf{\leftmark}}]
      {\fancyplain{}{\textbf{\thepage}}}

\cfoot{\fancyplain{\textit{Draft:} v\version, \draftdate}
                  {\textit{Draft:} v\version, \draftdate}}
\addtolength{\headheight}{1.6pt}
\usepackage{jpgref,varioref}
\usepackage{shortvrb}
\MakeShortVerb{\"}

% get bold tt
\DeclareFontShape{OT1}{cmtt}{bx}{n}{
      <5> <6> <7> <8>  cmbtt8
      <9> cmbtt9
      <10> <10.95> <12> <14.4> <17.28> <20.74> <24.88> cmbtt10}{}

\newcommand*{\prg}[1]{\textsf{#1}}		% for program names
\newcommand*{\cmd}[1]{\texttt{#1}}		% for cmd names
\newcommand*{\env}[1]{\textsf{\bfseries #1}}	% enviornment vars
\newcommand*{\file}[1]{\texttt{#1}}		% for computer files
\newcommand*{\scare}[1]{{``}#1\/{''}}		% scare quotes
\newcommand*{\nt}[1]{\textit{#1}}		% new term
\newcommand*{\opt}[1]{\texttt{#1}}		% option
\newcommand*{\btitle}[1]{\emph{#1}}		% titles of books
\newcommand{\dram}{\prg{distrat}/\prg{askmap}\xspace}	% package name
\newcommand{\MG}{M\&G95\xspace}			% Markoczy Goldberg 1995
\newcommand{\dash}{---}

% lists for option lists  % see Companion page 64 for this
\newlength{\lentrylen}
\newcommand{\lentrylabel}[1]{%
   \settowidth{\lentrylen}{#1}%
   \ifthenelse{\lengthtest{\lentrylen > \labelwidth}}%
      {\parbox[b]{\labelwidth}{\makebox[0pt][l]{#1}\\}}%   % term > labelwidth
      {#1}%						   % term < labelwidth
   \hfil\relax}
\newenvironment{optlist}[1][35pt]
    {\renewcommand{\entrylabel}{\lentrylabel}\begin{entry}[#1]}
    {\end{entry}}
\newenvironment{entry}[1][35pt]
  {\begin{list}{}%
    {\renewcommand{\makelabel}{\entrylabel}%
      \setlength{\labelwidth}{#1}%
      \setlength{\leftmargin}{\labelwidth}%
      \addtolength{\leftmargin}{\labelsep}}}
  {\end{list}}
\newcommand{\entrylabel}[1]{\mbox{\textsf{#1:}}\hfil}

\title{The \dram suite of programs for cause map analysis\\
   A user's guide}
\shorttitle{\dram user guide}
\author{Jeff Goldberg}
\shortauthor{J. Goldberg}
\draftinfo{DRAFT version~\version: \today}
% \date{\draftdate}

\begin{document}
\maketitle
\thispagestyle{empty}
\begin{prepage}
  \begin{abstract}
    This report documents the use of the programs referred to by
    \citeasnoun{MarkoczyGoldberg95:JOM} for causal map elicitation
    and analysis.  It only documents the software and not the method.
    Only the cause mapping analysis software is discussed and nothing
    is said about statistical analysis.
    \emph{This is still incomplete.}
  \end{abstract}
  \begin{egotrip}
    \textbf{Jeff Goldberg} is the author of the software this report
    documents, and has contributed to the method that the software
    is designed to support.  He has a bachelor's degree from
    the University of California, Santa Cruz in Linguistics, and
    was employed by the Hungarian Academy of Sciences
    Research Institute of Linguistics and the Theoretical Linguistics
    Program of the E\"otv\"os University (ELTE), Budapest when most
    of this software was written.  He is currently employed by the
    Computer Centre at Cranfield University.  He can
    best be reached as \textsf{J.Goldberg@cranfield.ac.uk}.  His
    PGP key finger print, for those who prefer secure Email, is
    \textsf{4E 33 96 71 1E 98 E7 93  20 E2 48 6D 52 A1 71 E0},
    and the key is available from keyservers and other places.
  \end{egotrip}
  \begin{acknowledgements}
    Your name here if you give me feedback.  If you do, please
    specify that your comments are on version~\version.
  \end{acknowledgements}
\end{prepage}
\pagenumbering{roman}
\tableofcontents
\cleardoublepage
\pagenumbering{arabic}
\chapter{Introduction}

This manual describes the \dram set of programs used to perform
certain tasks useful for causal map manipulation and analysis.  These
programs perform many of the tasks described by
\citeasnoun{MarkoczyGoldberg95:JOM}\footnote{\label{fn:jai}%
  JAI Press has refused to grant the authors permission
  make a copy of that paper available electronically.
  Such an action is well within their right, but I do think that
  granting that permission would not cut into sales and would
  increase the number of citations of the article.  Moreover, their
  refusal may well affact decisions of where we submit things in
  future.}
(henceforth \MG).  This manual
does not tell you everything you need to know in order to use the
programs.  In addition to this manual you will need to read \MG (at
least pages 305--316).  This manual is not a guide, description or
tutorial on that method, but only describes the software.
For reasons discussed in section~\ref{sec:modularity}, it can't
even be a complete guide to using the software on your system, because
I don't know the details of your system well enough to document how
you should edit files, create directories, etc.  Beyond that it is
intended that this will be a complete and definitive guide to the software.

Throughout this guide, you will find that sometimes I use ``I'' and
``me'', while at other times I use ``we'' and ``us''.  The plural
refers to myself and L\'{\i}via Mark\'oczy as authors of \MG.  The
singular refers to myself, Jeffrey Goldberg, the author of the \dram
software and this manual.

\section{How to use this manual}\label{sec:howtoread}

Read everything in this chapter.  Read only those sections of
Chapter~\ref{ch:install} that you need to in order to install the
programs on your computer system.  You must read or already
know much of the material in Chapter~\ref{ch:general} in order
to understand the actual descriptions of the programs.  If you
make the mistake of trying to read the chapters on the particular
programs before reading Chapter~\ref{ch:general} you will be left
struggling.  While not every sub-section of Chapter~\ref{ch:general}
is strictly necessary, you can only determine that by reading the
first few paragraphs of each of those sections.

The chapters describing the individual programs then follow, with
the large and more useful programs listed first.  Do not
attempt to skip to those chapters without going through
Chapter~\ref{ch:general} first.  Chapter~\ref{ch:distrat}
describes the \prg{distrat} program which is used to calculate
the \nt{distance ratio} between cause maps.  Chapter~\ref{ch:askmap}
describes the \prg{askmap} program which is used to assist the
researcher in eliciting maps.  Chapter~\ref{ch:drsort} describes
a program (\prg{drsort}) that helps prepare the output data from
\prg{distrat} for input to statistical programs.  Chapter~\ref{ch:density}
describes a program which calculates the \nt{density} of cause maps.
Chapter~\ref{ch:iodeg} describes \prg{iodeg} which is used of
calculated \nt{in-degrees} and \nt{out-degrees} of maps.
Chapter~\ref{ch:avmap} describes the program \prg{avmap} which is
used to calculate the \nt{average} or \nt{central} map of a set of
maps.  Chapter~\ref{ch:ross} describes a program \prg{ross} used to
get \nt{Ross orderings}.  These chapters are then followed by
a number of appendices.  The only one that I will mention here
is Appendix~\ref{app:gnu} which you \emph{must} read if you distribute
this software or modified versions of this software.  Even if you
don't intend to distribute the software you might find the preamble
of the license agreement of some interest.

\section{Is this software for you?}\label{sec:foryou}

The method described in \MG is, in my obviously biased opinion,
the best method to date for doing solid work with cause maps.  I hope
that it will be built upon and will be usable.  It is, however,
a rather narrow method, applicable to a special class of problems as
far as I can see.  This software, other than its limited intrinsic interest,
is not of much use to someone not using the method described.  Thus,
this software probably doesn't meet your needs.  In order to save
you some trouble, I list a set of guidelines which will help
you determine how useful this software is likely to be.  If
you are someone who enjoys downloading, installing, and reading
about new software, then you don't have to worry about whether it
will be useful.  The list is a rough set of guidelines which should
be taken has indicators of the chance that you will be disappointed
instead of as hard and fast rules.

\begin{itemize}
\item Have you still not read \MG?

   The software and this documentation will little sense to you until
   you read that paper.  I have heard rumors that the software and
   documentation makes more sense if you not only read the paper, but
   cite it as well.

\item Do you expect the programs to tell you \emph{what} the
   differences are between CMs instead of merely \emph{how large}
   those differences are?

   If so, you will be disappointed and
   have probably not read \MG.\footnote{Or \MG isn't written as
    well as it should be.}
   Some of the simplier tools,
   such as \prg{iodeg} (chapter~\ref{ch:iodeg})
   and \prg{density} (chapter~\ref{ch:density})
   may be of some minor, almost trivial,
   assistence in that, but those calculations can easily be
   done by hand, and play a small role in the eventual analysis.

\item
  Have you collected data before you worked out how to analyize it?

  If so, unless you are extremely lucky and careful in your data
  collection, your data is probably nearly unusable.  Although
  some more textually based data analysis tools may be of some
  help to you, \dram probably is not.

\item
  Do you not really understand the method described in \MG?

  It is not essential to understand the statistical analysis portion
  of that paper (if you don't wish to attempt that sort of statistical
  analysis).  Also experimenting with the software can help
  you understand the method, but obviously you should never use
  a method you don't understand.

  The equation for the distance ratio there may look frightening,
  but contains nothing more that addition, multipication, division,
  and subtraction.
  Read through the text a few times slowly, maybe even with a couple
  very small maps as samples and doing the calculation by hand.  If you
  have questions or comments about the method, please get in touch
  with either of the authors of \MG.  Remember, we want you to use the
  method, otherwise we wouldn't have gone to such effort describing it.
  A step by step example of the calculation appears
  in \citeasnoun[Appendix~D]{Markoczy95:thesis}.
  It really isn't as hard as it may first appear and every effort
  has been made to make it less mysterious.

\item
  Do you want to impress with numbers?  Are you impressed by them?

  Explicit quantitative analyses are benificial when
  they are used to make things more clear.  They tell your
  readers \emph{exactly} what you did to get your results and
  lay all of the many assumptions that go into an analysis
  bare for all the world to see and evaluate.  Methods
  should never be used to obscure or intimidate.
  \citeasnoun{EdenJones84:JORS} warned of the ``dubious opaqueness''
  introduced by making analysis more technical than necessary,
  and it is a warning worth noting.

  However,
  it is worth noting that the fact that a method does something
  with a computer can sometimes help \scare{sell} a project to
  participants or funding bodies.  Computer tools can make a method
  appear more modern or even more \scare{scientific}.  But
  don't oversell things on that point; you only insult your
  participents, discredit yourself, and\dash which is far worse
  from my point of view\dash discredit the author of the software
  and authors of the method.

\item
  Do you just want to see what the program looks like because it's there?

  Enjoy!  But it doesn't look very impressive.  The only mildly
  impressive bits are deep inside anyway, which show up only with the
  speed at which it does to calculation for large numbers of maps.
  Most of the programs are rather simple, and could be (and, I suppose,
  were) written by a casual programmer.  The advantage of this programs
  is that they are here.  They are most certainly \emph{not} flashy.

\item
  Do you want to use this software as an alternative to or
  replacement of other cause map software?

  There is almost no duplication between what these programs
  do and what other cause mapping software does.  This is
  not due to some great originality on my part; instead it is
  a consequence of laziness:\footnote{Laziness is one
  of the three great virtues of a computer programmer.  The other
  two are impatience and hubris \cite[p.~426]{WallSchwartz91:Camel}.
  I hope that my programming reflects these.}
  If there already was software that did what I needed, why
  should I write something myself?  Furthermore, now that these
  programs exist, there is little reason for anyone else to
  write new programs that do the same.  Of course, someone
  could make improvements on these programs, and I hope
  that they will.  For the most part, there is very little
  overlap in utility between these programs and other
  cause mapping software that I am aware of.

  Information about other cause map analysis software is listed
  in Appendix~\ref{app:other}.  It may very well be that your
  disappointment in this package is matched by satisfaction with
  some of these others.

\item
  Are you unwilling to read computer documentation?

  This is not plug and play take you by hand software.  Not only do
  you need to learn what it does (by reading \MG) but how to use it.
  Fortunately, you don't need to read the entire set of instructions.
  You should read the first bits (this chapter and 
  parts of chapter~\ref{ch:general}) and then just the parts that you
  need.
 
  Still don't be put off by the length of this documention.
  Review section~\ref{sec:howtoread} to see how to make
  the most use (with the least time) of reading the manual.

  And if you get stuck don't be afraid to ask for help.  Both
  authors of \MG want to see the method used and improved.

\end{itemize}

Now that I have frightened you off, let me try to coax you back.
I obviously want to see this software used.  Otherwise I would not have
written instructions at all.  (I didn't need the documentation myself,
and I personally trained the person for whom I wrote the software.)
So do take the existence of the documentation as a sign that I
really do want to help you use it if it will be useful to you.

I have also helped people use it who (according to the list above)
are not ideal candidates for my full enthusiastic support.  I'm sure
that I will continue to do so.  Furthermore, it would be foolish of
me to state conditions (other than what appears in the License Agreement)
on the use or distribution of the software.  You may have a creative
use for that software which I could not anticipate.  In fact, I hope
that you do.

Finally, if it does turn out that the software really isn't for you for
any of the reasons listed above or any of the countless alternatives,
don't be afraid to let me know why.  Methods only rarely compete with
each other.  For the most part they are designed with specific for
specific projects and problems, and if one method is not appropriate,
it is no shame on the authors of the method.


\section{Legal matters}

I must state that the software and accompanying documentation is
distributed \emph{as is} with \emph{no warranty whatsoever}.  See
the Gnu Public License Agreement in appendix~\ref{app:gnu}.  Because
the software is distributed absolutely freely, I can afford neither
to make any commitment to maintaining it, nor can I afford liability
insurance.

The software\dash including the source code\dash is distributed
freely and may be further distributed or including in other software
which will be distributed according to the conditions of the GNU
General Public License version~2 or later (included in the
accompanying file \file{COPYING} and in appendix~\ref{app:gnu}).  The
preamble to the license agreement contains its rational.  I want the
software distributed without restriction.  However, to make sure that
all potential users and programers would have equally unrestricted
access, I need to ensure that no one else places restriction on the
software.  Therefore, I must maintain my copyright in order to impose
a restriction that no further restrictions be placed on the material.

Additionally, once a method, including that described by
\citeasnoun{MarkoczyGoldberg95:JOM}, is published it can be used
freely by the research community, in which there are proper
conventions for crediting those who have developed the method.  It is
professionally in the interest in the creator of a method to see it
as widely used as possible, and if the method can be more easily
employed with certain software, it is in the interest of the
developers of the method to make that software as freely available to
the academic community as the method itself.

I have been able to do that for the programs included.  However,
there are a number of statistical procedures described by \MG for
which no accompanying software is provided.  In some cases
proprietary software was used \scare{off the shelf}, while in other
cases I modified and extended code which is not in my power to
distribute.  I can bring my modifications under the terms of the GNU
General Public License Agreement, but they are too well integrated in
to code copyrighted by Numerical Recipes Software and can only be
distributed to those who have a license from Numerical Recipes
Software which automatically accompanies the purchase of the
outstanding book \btitle{Numerical Recipes in C}
\cite{PressETAL88:Book}.

\subsection{Trademarks}

Throughout this document and others distributed with the programs
I make use of various trademarks.  In particular, I refer to
 ``DOS'', ``MS-Windows'', ``Windows95'', which are trademarks of Microsoft;
 ``Unix'', which is a trademark of Unix Systems Laboratories\footnote{%
        USL with all assets (including trademarks) has recently been
        purchased by Novell and I can not longer keep track of who
        the trademark is registered to any more:
	AT\&T, USL, Novell, X/Open\ldots I have given up keeping track.},
        but I use it to refer to Unix-like systems, such as linux,
        FreeBSD, Ultrix, OSF, etc (all of which are trademarks themselves)
        as well as Unix proper as licensed by USL;
  ``Turbo C'', which is a trademark of Borland Inc.;
  ``\TeX'', which is a trademark of the American Mathematical Society;
  ``\LaTeX'', which, I believe, is a trademark of Addison-Wesley;
   ``PostScript'', which may be a trademark of Adobe Systems;
   ``PCL'', and ``Lasarjet'', which may be trademarks of Hewlett-Packard;
   ``PKzip'', which is also a registered trademark.

The above list is far exhaustive.   Other trademarks may be used in this
document; their use here and ommission from the above list should not
be taken in any which diminishes their status as trademarks.

\section{How to reach the author(s)}

\subsection{Reaching Jeff Goldberg}
You can reach Jeff Goldberg by Email at \textsl{J.Goldberg@Cranfield.ac.uk},
or by post.

\begin{verse}
Jeffrey Goldberg\\
Cranfield University Computer Centre\\
Cranfield University\\ Wharley End MK43 0AL, UK
%\\[4pt] put in FAX when I get it.
\end{verse}

I actively encourge people to use secure email for even mundane transactions
and for those who wish to use PGP\footnote{[I will need to add one pointer
to where to get information on PGP].}
my public key (which is available
via \path|http://www.cranfield.ac.uk/public/cc/cc047/|, or
\path|finger:goldberg@csli.stanford.edu|, or
\path|finger:goldberg@ny01.nytud.hu| or from any keyserver
\textsf{pgp-public-keys@keys.pgp.net}) my key fingerprint is
\textsf{4E 33 96 71 1E 98 E7 93  20 E2 48 6D 52 A1 71 E0}.

\subsection{Reaching L\'{\i}via Mark\'oczy}
L\'{\i}via Mark\'oczy can be reached by email at 
\textsf{L.Markoczy@Cranfield.ac.uk}.  Or
\begin{verse}
Dr.~L\'{\i}via Mark\'oczy\\ Cranfield School of Management\\
Cranfield University\\ Wharley End MK43 0AL, UK\\[4pt]
FAX +44 (0)1234 751-???; Tel: +44 (0)1234 751122
\end{verse}
Her PGP key finger print is 
\textsf{F9 81 94 D7 AE 06 1D C7  5A F6 F1 E2 39 12 7D 8F}


%%%%%%%%%%%%%%%%%%%%%%%%%%%%%%%%%%%%%%%%%%%%%%%%%%%%%%%%%%%%%%%%%
%%%%%%%%%%%%%%%%%%%%%%%%%%%%%%%%%%%%%%%%%%%%%%%%%%%%%%%%%%%%%%%%%
%%%%%%%%%%%%%%%                              %%%%%%%%%%%%%%%%%%%%
%%%%%%%%%%%%%%%   End introductory chapter   %%%%%%%%%%%%%%%%%%%%
%%%%%%%%%%%%%%%                              %%%%%%%%%%%%%%%%%%%%
%%%%%%%%%%%%%%%%%%%%%%%%%%%%%%%%%%%%%%%%%%%%%%%%%%%%%%%%%%%%%%%%%
%%%%%%%%%%%%%%%%%%%%%%%%%%%%%%%%%%%%%%%%%%%%%%%%%%%%%%%%%%%%%%%%%

\chapter{Obtaining and installing \dram{}}\label{ch:install}

\section{Obtaining \dram{}}\label{sec:obtain}

This reviews various ways of obtaining the software.  There is some
difficulty in working out the appropriate level of detail for this
chapter.  For those who are familiar with getting things over the
network, I can just tell you the short answer which is
\begin{itemize}
\item
    The Management archive
    \path|gopher://ursus.jun.alaska.edu| in the
   \textit{Management Related Software} directory.  Or
\item
    \path|ftp://ftp.cranfiled.ac.uk/public/cc/cc047/cause-mapping|
    which will always have the most up to date version.  Or
\item
    \path|http://www.cranfiled.ac.uk/public/cc/cc047/cause-mapping|\\
    which is just another way of getting to the same files.
\end{itemize}
In each case look for the file called \file{read.me} and read it
for further instructions.

For those of you who find the above cryptic, the rest of this section
should should help.  But keep in mind that I most certainly can't
tell you how to call up various internet tools at your site (ask for
help locally) nor even tell you which tool is best or what the best location
to go to is.  That depends on what tools you have available (and what
your preferences are) and where you are.

It is worth learning how to get things from the internet, even if the
\dram package is of very little utility to you, since so many other
more valuable things are available freely often under the terms of
the GNU license agreement in Appendix~\ref{app:gnu}\@.  You should
ask your local computer support people for help, since they can show
you how best to do this.  Have this chapter ready and make an
appointment with some computer support people (or, better yet, with a
colleage who knows about these things) and do it.

\subsection{How \emph{not} to get \dram}\label{sec:hownot}

Do \emph{not} ask me to send you diskette.
Do \emph{not} send me a diskette.

With that said, let me list exceptions.  I may be persuaded to
arrange to have a disk sent to you if
\begin{itemize}
\item	Your site doesn't have any internet access.  You
	should complain \emph{loadly} to the university administration
	if this is really the case.
\item	You have given this a try, but failed.
\item	You can persuade me that you have a project in mind which
	requires this software and is likely to lead to publication.
	That is, the software really ``is for you''
	(see section~\ref{sec:foryou}).
\end{itemize}

\subsection{How to get \dram using gopher}\label{sec:gopher}

If it is easier for you to reach the UK than Alaska from your site,
please look at section~\ref{sec:cranfield} (Getting the files in Europe)
as well as this section.

If you are familiar with the Management Archive
  (\path|gopher://ursus.jun.alaska.edu|)

\begin{enumerate}
\item
    Get the files from the Management Archive.
    The programs are in a clearly labeled directory/menu in the
    ``Management-Related Software'' directory.

\item
    Look for a ``Read Me'' file which will tell you which files you
    need to get.  Which files you need depends more on your what
    kind of archiving and uncompressing software you have, and
    whether you want the DOS executables.  Look at the
    sections \ref{sec:zip}~and~\ref{sec:tar.gz} for some information.
    Note that some gopher, ftp, or www programs which you might run
    may decompress some of the files for you automatically.

\end{enumerate}

If you are not familiar with the Management Archive

\begin{enumerate}
\item Jim Goes maintains an electronic archive of management related
     material.  It is freely available over the Internet.  It includes
     many useful things including working papers.  (You can also put
     your own working papers there.)

\item The archive is accessible via the \prg{gopher} program (or any
      WWW browser).  Your department's,
      school's or university's computer will have the program.  You may
      need to ask your computer support staff about this.  But it should
      be on some machine that has internet access.

     If you have World Wide Web (WWW) software (such as \prg{Mosaic},
     \prg{Lynx}, or \prg{NetScape}) you will not need a specific
     gopher program, since gopher is subsumed by WWW.

     If your cite really doesn't have gopher, ftp or a WWW browser at all,
     then your system administrator will have to install one.  All it
     requires is that you have internet access.  If you don't have internet
     access you will have to find somoene who does.  You should cetainly
     make sure that your institute gets internet access.\footnote{%
	If your institution is based in a region where internet access
	is limited (usually due, not to poverty, but to the obstinence of
	national telecoms monopolies and governments which protect them)
	then I will send you a disk if you promise to make some statement
	in writing to the head of your instituton or to a relevant
	government ministy requesting that internet access be expanded.}

\item   Do feel free to ask your computer support people for help with some
     steps in the processes.  You may be able to figure things out
     after spending a lot of time that could be saved just by asking.

\item You will need to point gopher to the management archive which is
     at \path|ursus.jun.alaska.edu|.

     I cannot tell you how to do that, since
     gopher programs differ.

\item Once connected you should get and read the information available
     there by the menu.  Note that menu items that end with the
     character "/" have submenu's.  Also note that you are reading
     information off of a machine in Alaska.  Depending on where you
     are you will require various amounts of patience.

\item The programs are in the ``Management-Related Software Archive''
     section of the Management-Archive.  You will see the menu
     item clearly.

\item Once you get into the menu with the distrat files you will need to
     save the files to your machine.  On many gophers you just type
     `s' on a menu item and it will copy that file to your machine.

\item  Once you have learned how to reach the Management Archive, you
     naturally may wish to do so again.  You are encouraged to do so.
     In the main directory of the Archive there are files that tell
     you more about it, and which tell you how to have your papers
     placed on the archive.  It is too good a tool to pass up.
\end{enumerate}

\subsection{How to get \dram with FTP}

You only need to read this section if there was no gopher or WWW to use,
or if you prefer \prg{ftp} in which case you probably don't need
to read this section either.

First you will need a machine with Internet access and the \prg{ftp}
(File Transfer Protocol) program.  Any machine that has an internet
connection will have \prg{ftp}.  If you need help finding \prg{ftp},
ask your computer support staff.

Connect with ftp to ursus.jun.alaska.edu with the command
\begin{alltt}
     ftp ursus.jun.alaska.edu
\end{alltt}
Of course the way that you call up ftp on your machine may differ.

You will get some welcoming message and will then be asked something like:

\begin{alltt}
    username (yourname):
\end{alltt}
You should type in \texttt{anonymous}
Next you will be told ``Anonymous login okay.  Use complete E-mail address
as password''

It will ask \texttt{Password}
Do not type in your real password, do type in your E-mail address.

After this you will get some more welcoming messages.  You may have
to wait a few seconds (or not so few if you are connecting from a bad
subnetwork) for responses from \path|ursus.jun.alaska.edu|.

Your prompt from \prg{ftp} may look like

\begin{alltt}
ftp>
\end{alltt}

But this depends on what system you are using.  But to change directories
with ftp on the Management Archive you use the command "cd" to ftp
and the directory separator is "/" and not "\".

You will need to change directory to \file{/pub/distrat}.  You can do this in
stages by 
\begin{alltt}
ftp> cd pub
\end{alltt}
and 
\begin{alltt}
ftp> cd distrat
\end{alltt}

At any point you can type either \cmd{dir} or \cmd{ls} to get a list of files
and directories in your current directory.  You can type \cmd{pwd} to
to see what your current directory on the Archive is.

Once you are there you should get the files with the get command.
All of the files, with the exception of \file{read.me} should be
transferred in \nt{binary} mode.  Type the command \cmd{bin} to ftp
to set up binary mode transfer.  (Type \cmd{ascii} to set up
ASCII mode transfer).  So you might do something like the following
(You only type what is after the \texttt{ftp>}.  The rest is produced by
ftp).

\begin{alltt}
ftp> ascii
Type set to A
ftp> get read.me
local: read.me  remote: read.me
Opening connection for \ldots
\ldots Transfer complete   (\ldots Kbytes/Sec)
ftp> bin
Type set to I
ftp> get drdist.zip
\ldots
\end{alltt}

You leave ftp with the either \prg{quit} or \prg{bye}.  Of course you can
use the opportunity of your connection to look around and see what
other good things you may wish to get from the Management Archive. 

\subsection{Getting the files from Cranfield, UK}\label{sec:cranfield}

The files are also archived at Cranfield University in the UK.  They
can be retrieved either with ftp to ftp.cranfield.ac.uk in the
directory \path|/public/cc/cc047/cause-mapping/| or by using a World Wide
Web browser to
\path|http://www.cranfield.ac.uk/public/cc/cc047/cause-mapping/|

\subsection{Getting the files from others}

Just because I (Jeff Goldberg) am too lazy to send you a diskette
with the software (see section~\ref{sec:hownot}), doesn't mean that
somebody else won't.  If you for some reason can't get access to
gopher or ftp or just don't want to, you can try to persuade a friend
to do that for you and give you (or send you) as diskette.

There is nothing barring someone from going into the business of
selling that kind of activity as a service as long as they give you
all of the files that you have a right to as part of the system, and
inform you of your rights under the terms of the General Public
Software License Agreement in Appendix~\ref{app:gnu}\@.

\subsection{Which files}

There is a text file called ``read.me''.  Get and read that for further
instructions.  It will more accurately reflect the situation on the
site that you are getting the files from than this documentation can.

If you do not need DOS executables, then you will only need the
source and documentation compressed (tar-ed and gzip-ed) archive
called something like \file{drdist.tgz}  for 
``Distance Ratio Distribution (tar gzip)''

If you need DOS executables you need the file \file{drdist.zip}.

The ASCII files archived in \file{drdist.zip} have DOS style newlines
(0x0d0a) while the ASCII files archived in \file{drdist.tgz} have
Unix style linefeeds (0x0a).

MSDOS users may also need the two programs \file{tar.exe} and \file{gzip.exe}
or \file{unzip.exe} to uncompress and extract from the archives.  These
are either immediately present or the readme file will state where they
are.  The developers of those programs have placed them under the
terms of the GNU software license agreement (Appendix~\ref{app:gnu}\@).
The shareware program \prg{PKunzip} can be used instead of \prg{unzip}
if, for some reason, you prefer paying for software instead of getting
it for free.\footnote{\prg{PKunzip} is an excellent program, but \prg{unzip}
is better, free, available with source, and works on many platforms.} 


%%%%%%%%%%%%%%%%%%%%%%%%%%%%%%%%%%%%%%%%%%%%%%%%%%%%%%%%%%%%%%%%%%%
%%%%  End of Obtaining secton   %%%%%%%%%%%%%%%%%%%%%%%%%%%%%%%%%%%
%%%%%%%%%%%%%%%%%%%%%%%%%%%%%%%%%%%%%%%%%%%%%%%%%%%%%%%%%%%%%%%%%%%
%%%%%%%%%%%%%%%%%%%%%%%%%%%%%%%%%%%%%%%%%%%%%%%%%%%%%%%%%%%%%%%%%%%
%%%%%%%%%%%%%%%%%%%%%%%%%%%%%%%%%%%%%%%%%%%%%%%%%%%%%%%%%%%%%%%%%%%
%%%%%%%%%%%%%%%%%%%%%%%%%%%%%%%%%%%%%%%%%%%%%%%%%%%%%%%%%%%%%%%%%%%

\section{Unpacking the software}

How the software is installed depends on the system that you wish to
install it on.  The programs are designed to work on a variety of
systems, but require some very specific procedures for the very
specific system you have.  If you are not comfortable with the steps
that follow, you should ask your computer support people for help.
Show them the installation section of this document and ask for their
assistance.   You could even give them advance warning by sending
them a copy of this chapter prior to meeting, so they know exactly
what tools will be needed.

\subsection
   {Unpacking the software from the \file{.zip} files}\label{sec:zip}

If you got this far then you have probably unpacked the things
already, since this document, when it is completed, will be in the
distribution.  However, what you probably did to unpack the
distribution is

\begin{verbatim}
c:
cd \
a:unzip a:drdist.zip
\end{verbatim}
which created a lot of files in the directory "C:\distrat".  Note
that if you use the \file{.zip} file on Unix you should use the \opt{-a}
option to \prg{unzip}.

\subsection
   {Unpacking the software from the \file{.tgz} files}\label{sec:tar.gz}

Note that different archives may use different names depending
on what they can handle and when they think users can handle.  A
``gzipped tar file'' really should have the extension \file{.tar.gz},
but some systems don't take that, so this is alternatively called
\file{.tgz} or \file{.taz}.  After un-gzipping, the file on all
systems should be \file{.tar}.

If you have gotten far enough to read this, you have successfully
unpacked the files.  However, as a reminder of what you did, you
copied the files \file{drdist.taz} (or \file{drdist.tar.gz}) to
some place on your machine, and ran
\begin{verbatim}
  gzip -d drdist.taz
\end{verbatim}
or
\begin{verbatim}
  gunzip drdist.taz
\end{verbatim}
which uncompressed the file \file{drdist.taz} and replaced it with
a larger file called \file{drdist.tar}.  Your next step was to issue
the command
\begin{verbatim}
  tar -xvf drdist.tar
\end{verbatim}
which extracted all the files from the archive, building several directories.

If you do this on DOS, you may have to give special options to the
\prg{tar} to make sure that ASCII files come out okay.

\subsection{Directory structure after unpacking}

After unziping (or untaring) you should have the directory
structure listed below.  Note that I use "/" instead of "\" to
list directory separators here.
\begin{center}
\begin{tabular}{>{\ttfamily}lp{7.5cm}}
  distrat		& The main directory, containing a readme, and
			  some addministrative files.\\
  distrat/source	& The source files (C programs, makefiles, etc).\\
  distrat/bin		& DOS AT executables.  For 286 and higher (they
			  work fine on 386s, 486s, an Pentiums).\\
  distrat/386bin	& DOS 386 (or higher) executables.  These
			  can make use of all of your machine's free memory.
\end{tabular}
\end{center}

\section{The documentation}

Save the documentation (the files in the \file{doc} directory).
Even though you may not need certain programs at the moment,
your needs may change.  This documentation is you are now
reading is intended to replace most of the separate files in the \file{doc}
directory, but if they are still included in your distribution,
then they are meant to be.

\subsection{Printing the new documentation}

The full documentation you are now reading can be printed out on
either a PostScript printer or a PCL (HP Laserjet compatible) printer
without requiring any text processing software at all.  All you need
to do is send either \file{drdoc.ps} or \file{drdoc.hp} directly
to your printer, without loading and ``printer drivers''.   The
full documentation is more than~\ref{LastPage} pages long.\footnote{%
  My apologies for being so long winded, but it is difficult
  to write a document that contains all the information that
  each person needs.}

Because I don't know the details of your system, it is difficult for
me to specify exactly what steps you need to take to print.  However,
the documentation is in such a form that you need no special software
(e.g., ``Word'', ``WordPerfect'', ``\TeX'', etc), nor printer
drivers to print out the documentation.  You do need one of
two very common types of laser printers.  The documenation is
distributed in four forms listed in order of decreasing utility in
table~\ref{tab:docfiles}

\begin{table}
\begin{center}
\begin{tabular}{>{\ttfamily}lp{7.5cm}}
     drdoc.ps	&	PostScript file (300dpi)\\
     drdoc.hp	&	PCL (HPlaserJet) file\\
     drdoc.dvi	&	If you know that this is, use it, otherwise,
			you can very safely delete it.\\
     drdoc.tex	&	 LaTeX source (not really usable since
			 many specialized macros are not included)
\end{tabular}
\end{center}
\caption[Forms of the documentation files]
  {Forms of the documentation files in order of decreasing utility}
  \label{tab:docfiles}
\end{table}

Since I have received few requests for the software from North America,
these documents are produced for A4 paper.  North Americans can
still print them out on letter paper, but the margins will be funny.
Also, I can easily produce letter sized versions if you would rather
have that, but I don't want to clutter up the distribution files
with many versions of the same file.

There are (a bit more than) three things you need to figure out
at your site for printing:
\begin{enumerate}

\item    What Operating system are you using?
  
         MS-Windows and Windows95 are DOS as far as this documentation
	 is concerned, but the commands listed are still why you type
	 to a DOS prompt, which you can get at by opening a DOS window.

         (I will discuss only
	 DOS and Unix below.  If someone can provide information for
	 Macs, I would be more than happy to include that here.  Also,
	 I will be assuming that Novell users will be able to more
	 or less use the DOS instructions.


\item    What kind of printer do you have access to?

         You will need either a printer
	 which understands PostScript (A printer language) or
	 which understands PCL (another printer language), which
	 is the language used by HP LaserJets and compatibles.  If
	 you have a choice between the two, use the PostScript printer.
	 Note that some printers accept both and automatically detect
	 which kind of data it is getting.

         If you only have a postscript printer, then you can remove
         the .hp file, and if you only have access to a PCL printer
	 you can remove the .ps file.  

\item\label{en:print:name}
         What is the name (port) to print on to get to that printer?

	 For DOS users, this will be something like \file{lpt1:} or
	 \file{lpt2:}.  For Unix the names are far more arbitrary,
	 like, \file{fred\_ps}, or \file{Bldg63\_P1}.

         For the rest of this section,
         I will use the following for my examples
	 \begin{center}
	 \begin{tabular}{ll}
	   \file{lpt1:} & DOS PCL printer\\
	   \file{lpt2:} & DOS PostScript printer\\
	   \file{escher\_ps} & Unix PostScript printer\\
	   \file{durer\_ps} & Unix PCL printer
	 \end{tabular}
	 \end{center}
	 \emph{You will have to substitute
	 these with the correct values (ports or names)
         for your printer and system.}
	 
\item	 DOS users may encounter other problems which I can only
	 half anticipate, since there are too many ways things can
	 be set up at your end.
\end{enumerate}

\subsubsection{Printing \file{drdoc} on DOS}

First, let me remind you that you should \emph{not} try to load
either \file{drdoc.hp} or \file{drdoc.ps} into some sort of text
processor or ``printer driver''.  These files have already been
prepared for common types of printers and need no further preparation.
They are already in the form that the printer expects them in.

First go into the \file{doc} subdirectory of the \file{distrat} directory.
If you did everything as above with unpacking that will be
\path|C:\DISTRAT\DOC\|.  The command "cd \distrat\doc"
ought to get you there.

To print to a PCL printer on \file{lpt1:} 
\begin{alltt}
C:> copy/b drdoc.hp lpt1:
\end{alltt}
Note the `\texttt{/b}' option for the copy command.  This is
necessary for printing PCL files.  Do \emph{not} use the
DOS \prg{print} command; it will not work.

To print to a PostScript printer on \file{lpt2:} 
\begin{alltt}
C:> copy drdoc.ps lpt2:
\end{alltt}

\paragraph{Special Problems with DOS}

Depending on how your system is designed to talk to the printer, you may
get an error message every few pages (depending on the size of the printer's
memory).  Keep hitting ``retry'' when prompted to.  If this doesn't
work, please ask someone local for help.

\subsubsection{Printing \file{drdoc} on Unix}

To print on a PostScript printer called \file{esher\_ps} use
\begin{alltt}
\% lpr -Pescher_ps drdoc.ps
\end{alltt}

If you have a PCL printer called \file{durer\_laser} use
\begin{alltt}
\% lpr -Pdurer_laser drdoc.hp
\end{alltt}

\subsubsection{If you have problems}

If you have problems printing this, take the files, and a printed
of what you are now reading to someone who knows your
system for more help.  If that person needs to contact me with
more questions they should feel free.  I may be able to help
you remotely, but too much of how to print is specific to how
your system is set up.  If you do wish to contact me for
help, please be able to answer the three questions listed above
about your printing system.  There is no way that I can answer those
questions remotely.


\section{Installation}

After you have unpacked things and taken a look at the documentation,
you just need to put files in the correct directories.  If you don't
wish to (or need to) compiling things yourself, installation is easy.

\subsection{Installing on DOS}

If you use the DOS executables from the \file{drdist.zip}
file then your installation
is easy.  Just place the executable files in some directory which is in
your \env{PATH}.\footnote{%
 The \env{PATH} environment variable specifies
 a list of directories in which programs are searched for when you type
 a command name.  It is usually set in the \file{autoexec.bat} file with
 the command \cmd{path}.}
If you plan to use the language features of the \prg{askmap} program, be
sure to place copy the \file{.aml} files from the \file{SOURCE} directory
in a safe place.  See the discussion in section~\ref{sec:amlfiles} for
more details.

If you do wish to compile the programs yourself
see section~\ref{sec:compile} for instructions.  If you don't know what
this means then it very unlikely that you should recompile.

The distributed executables in \file{bin}
are for 80286 or higher, running DOS 3 or
higher.  If you need executables for an XT (8088), please contact the
author.  If you are running on an 80386 or higher and you are finding
yourself running up against memory limitations or need an extra bit of
speed, you can use the executables in \file{386bin}.  Note that
to run those programs, the file \file{go32.exe} must be in your
path.  Also note that the \file{.aml} files are not duplicated
in the \file{386bin} directory, so you will need to copy them
from the \file{bin} directory or the \file{source} directory.
Once you have decided which you want to use, you may wish to
delete the other executables to save space.

\subsection{Installing on Unix}

Make sure that if you get the \file{.zip} file that you use the 
\opt{-a} option to \cmd{unzip} when you unzip the archive, because
files in that archive have DOS style newlines.  Alternatively, use
the \file{.tar.gz} which has Unix style newlines in the first place.
For Unix, executables are not distributed, and you need to compile
the files yourself as described in the next section.

\subsubsection{Compiling the files}\label{sec:compile}

In the \file{source} directory you will find a number of files called
\file{makefile.\textit{xxx}}.  The ``\texttt{\itshape xxx}''
indicates what system it was desigend for.  You should identify which
is the closest to your system.  (\file{makefile.tcc} and
\file{makefile.djp} are for DOS, so it is unlikely that you need
those.)

If you have \cmd{gcc}, try \file{makefile.gcc} first:
\begin{verbatim}
  cp makefile.gcc Makefile
  make all
\end{verbatim}

\subsubsection{Why aren't Unix executables distributed?}

The form of a binary executable is a function of two things:  The hardware
of the machine it runs on, and operating system it runs under.  Since
Unix runs on so many different kinds of hardware, and for the same hardware
there are often a variety of different Unices which have to be considered
different operating systems as far as binaries are concerned, it would be
difficult to produce binaries for so many different compinations of systems.
Furthermore, you have better access to the hardware and operating system
you will be using than I do; so, naturally, it is actually easier for
you to compile then for me.  If you encounter difficulties on you system,
you should let me know.

\subsubsection{Placing files}

If the make went okay, you should have a number of binary executables.
You can place them in a directory in your path (say \file{\$HOME/bin}), run
the shell command \cmd{rehash} if your shell needs that, and you are ready
to go.  You should also see the section~\ref{sec:amlfiles} about the
placement of the \file{.aml} files.

The binaries to install are
 \prg{askmap},
 \prg{avmap},
 \prg{density},
 \prg{distrat},
 \prg{drsort},
 \prg{iodeg},
 \prg{ross}.

The program \prg{cmtran} is utterly useless as a program, so should
not be installed, but is provided as a template.

\subsection{More on compiling the software}\label{sec:make}

This section provides more details about compiling the software.
Chances are you don't the material in this section, but it is
provided in case you run into difficulties.  If might be useful
to someone who is helping you compile things.

If you are not installing executables already created for your system,
either because the distribution does not have such executables, or
because you would like your own specially configured version, or
for safety reasons (A quick glance through the source files will
convince you there are no viruses, etc), or any number of these reasons,
you will need to compile the sources.

If you are familier with \prg{make} and \prg{C} then you should just
need to look at the section~\ref{sec:pickmake}, and in the event of
errors section~\ref{sec:makeprobs}.

\subsubsection{What you need}

In addition to the files in source directory for this distribution you
will need a
\begin{itemize}
\item
    C compiler that does ANSI for your type of machine.  gcc is
    recommended.

\item
    A linker/loader.  (If you have a C compiler you will probably
    have a loader).

\item
    The \prg{make} utility.  (Not actually necessary, but this documentation
         assumes that you have it.)
\end{itemize}
Note you won't need these tools to run the executable programs,
only to make them.  So, if you don't have a C compilier you could
compile the code on a machine that does (as long as it produces the
right kind of executables for your machine).

\subsubsection{Picking a Makefile}\label{sec:pickmake}

There are several makefiles provided, and with luck, one will work
for your system without modification.  The file \file{makefile.dst} is
the general one that all of the others are modelled from.  It will
\emph{not} work on any system.  The make files provided are listed
in table~\ref{tab:makefiles}
\begin{table}
\begin{center}
\begin{tabular}{>{\ttfamily}lp{7.5cm}}
	Makefile.tcc &	TurboC on DOS.  Use this as a model for
			other compilers on DOS.\\
	Makefile.dgp &  For DOS \prg{gcc} (\prg{djgpp}).  This file has
			an extra layer for the \prg{coff2exe}
			stage needed by \prg{djgpp}. \\
	Makefile.gcc &	Standard Unix with \prg{gcc} (Best and cheapest
                        C compiler around) \\
	Makefile.unx &	Standard Unix, but using \prg{cc}\\
	Makefile.dst &	Template \file{makefile}.  Will \emph{not work}
			on any system.
\end{tabular}
\end{center}
\caption{Makefiles to choose from}\label{tab:makefiles}
\end{table}

What you should do, copy the file that seems the most reasonable to
you to the file \file{Makefile}:

\begin{alltt}
  cp Makefile.\textit{xxx} Makefile
\end{alltt}
or
\begin{alltt}
  COPY makefile.\textit{xxx} makefile
\end{alltt}

And just try it the command \cmd{make all}

If all goes well, you should have the eight executable programs:
\prg{askmap}, \prg{avmap}, \prg{cmtran}, \prg{density}, \prg{distrat},
\prg{drsort}, \prg{iodeg}, \prg{ross}.  On DOS these will have
the extension \file{.exe} on their names.

You can copy these to where every you want them, or you can use
\cmd{make install}

But, before using \cmd{make install}, you should look at the \file{Makefile}
and see that \texttt{DESTDIR} is set up the way you want it.  (and that
the directory exists on the system).  So you might have to
edit the \file{makefile} for \cmd{make install}.  So either use your
``copy'' command to copy the files by hand, or read the next section
to learn more about \prg{make}.
  
\subsubsection{Can't find \file{unistd.h}?}

If, when trying to compile on Unix, you get an error about not being able
to find a file called \file{unistd.h}, then you will need to
modify the makefile that you are using.  Look for "HAVE_UNISTD_H"
in the file, and put a "#" character at the begining of the line.
Also, let me know, so that I can list in this documentation
what systems appear not to have \file{unistd.h}.  More information
about how \prg{make} works, and what a \file{makefile} really
is is listed in section~\ref{sec:moremake}.

\subsubsection{Lib warnings from \prg{tcc}}

The first time you compile using \prg{tcc}, you will probably get
warning messages from \prg{tlib} about modules not being in
the library.  Don't worry about this at first.  Those warnings
are just \prg{tlib}'s way of saying that the modules are being
added to the library for the first time.

\subsubsection{Other problems compiling}

The number and range of difficulties that you could encounter when
compiling makes it difficult for me to list out everything.  But
in addition to those listed above, you may need to refer to
section~\ref{sec:makeprobs} for other half anticipated problems.
If my programs are written correctly,
then the only thing that needs to be modified are the \file{makefiles}
or things about your system.\footnote{%
  If my programs are not written correctly (beyond the ``limits'' and ``bugs''
  listed in most chapters), then if I could tell you how to fix it, I
  already would have fixed it.}
If you do have problems compiling, please let me know.  In many cases,
someone at your site is more likely to be able to help then I will,
but you can it would still be useful for me to know (especially if there
is something in the source files that need to be changed).

\subsection{About \prg{make}}\label{sec:moremake}

If you are unfamiliar with \prg{make}, it is worth understanding basically
what make does.  If you do know about \prg{make}, then just skip this
section.   If you are just installing binaries or the quick installation
instructions worked for you, then you certainly don't need to read
this little introduction to \prg{make}, and you probably have a
better use of your time.  This section is to give you some
conceptual background if you actually do need to modify the makefiles.

\prg{Make} reads a special file (called a \file{Makefile})
which lists what files
depend on what, and how to make the former.  For example, suppose
that you worked in a text editor that created files in a particular
form (lets say .XXX), but for other purposes you sometimes needed the
file in RTF form (rich text format).  Suppose that your editor didn't
have a way of saving in RTF, but that you had a program, called
xxx2rtf which did the translation.  If this case, the .rtf file (the
target) depends on the .xxx file, and the makefile would have a rule
like

\begin{alltt}
foo.rtf:        foo.xxx
                xxx2rtf foo
\end{alltt}

If this were the case, then when you typed the command, \cmd{make
foo.rtf}, then make would check to see if the file \file{foo.rtf}
already exists, if it does it would check to see whether the file
\file{foo.rtf} as been modified more recently then the file \file{foo.xxx}.
If so, make would do nothing because foo.rtf is up to date.

If, on the otherhand, \file{foo.xxx} has been modified more recently (or is
not up to date), then make would run the command  \cmd{xxx2rtf foo}.
If \file{foo.xxx} depends on other things (as possibly stated elsewhere in the
makefile), then make would first try to ``make foo.xxx'' and only after
\file{foo.xxx} is up to date would it run the command to make \file{foo.rft}
from \file{foo.xxx}.

An almost real life example: I use the typesetting system \TeX.  The
program \TeX\ reads in a file which is usually created by a person
and produces something called a \file{dvi} (DeVice Independent) file.
To print the \file{.dvi} file, one needs to translate from that to
the kind of file that the printer wants.  Let's say an \file{.hp}
file for printers that use PCL (Hewlett-Packard compatible Printer
Control Language).  In this situation one could have a makefile with
the following information:

\begin{alltt}
foo.dvi:        foo.tex
                tex foo.tex

foo.hp:         foo.dvi
                dvi2hp foo
\end{alltt}

So if I edit \file{foo.tex} and want to produce the \file{.hp} file,
I would just type \cmd{make foo.hp}.

Things can get more complicated.  Sometimes I don't what to type in
the \file{tex} file directly, but have it produced by a program that reads
something that I do type in and translates that to \TeX.  There is
a program that will automatically put in the \TeX\ commands for Hungarian
hypenation so the user doesn't have to use a special version of \TeX\
for Hungarian.  This program, \prg{hion}, takes files \file{.hun} and makes
\file{.tex} files.  In this case it is the \file{.hun}
file that the human edits.
So the makefile would look like:

\begin{alltt}
foo.dvi:        foo.tex
                tex foo.tex

foo.hp:         foo.dvi
                dvi2hp foo

foo.tex:        foo.hun
                hion -TEX foo.hun
\end{alltt}

Of course A \TeX\ file can input other files, so sometimes the \file{.dvi}
file depends on more than one file.

\begin{alltt}
foo.dvi:        foo.tex table1.tex figure1.ps appendix1.tex
                tex foo.tex
\end{alltt}

And there could be all rules about making the things that \file{foo.dvi}
depends on.

Two more things about \prg{make}.  First it is possible to teach make how
to make things based on the file extensions instead of the full file
names.  So for example, it would be possible to tell make that any
file \file{\textit{name}.dvi} can be made from any
file \file{\textit{name}.tex} by running \cmd{tex \textit{name}.tex}.
I won't explain how.  Some of these sorts of rules are
``builtin'' to make.

The second other thing about \prg{make}, is that you can define variables

\begin{alltt}
RENAME=ren
\ldots

             $(RENAME) foomain.dvi foo.dvi
\end{alltt}

Lines in a make file that begin with a \texttt{\#} are not processed by make;
So a makefile could have

\begin{alltt}
#### use the ren for DOS
RENAME=ren
#### or mv for unix
#RENAME=mv
\ldots
        $(RENAME) foomain.dvi foo.dvi
\end{alltt}

If none of the Makefile templates that are provided with this
distribution work exactly for your system, then you will have to
change some of these sorts of things for your system, eg, from the
above to
\begin{alltt}
#### use the ren for DOS
#RENAME=ren
#### or mv for unix
RENAME=mv
\end{alltt}

See the difference?  (the second uses \prg{mv}, while the former will use
\prg{ren}).

\subsubsection{If there were problems with \prg{make}}\label{sec:makeprobs}

What you need to do depends on the program, so you will need to read
the output of make carefully.  Note that \prg{make} also displays the
commands that it calls.  This way you can see how far make got.

\begin{itemize}
\item Error type:   Out of memory

    You are obviously using DOS.  See the section on special
    instructions for DOS (section~\ref{sec:dosprobs}).

\item Error type:  blah blah Command not found
    
    If this is something like gcc command not found, then it is likely
    that the named C compiler from the makefile does not exist on
    your system.  The makefile has a section for selecting the compiler.
    You will need to make changes in that section.
    
    If you are on Unix you can safely use "CC=cc".  On DOS you will
    need to know what the name of your compilier is.  You may need to
    change many things in the DOS version if you are not using tcc.   
	
\item Error type:  file \file{limits.h} not found

    This error might happen during a compile (most likely when
    compiling \file{rossify.c}).  It means that you don't have
    a properly configured C compiling environment.  You can get around
    this particular problem by setting "-DNOLIMITS" in the ``Predefineds''
    section of the Makefile.  But chances are, if this has gone wrong
    other things will go wrong as well.

    You should safely be able to restart the make, without having
    to remake the other things already made.

\item Error type:  parse error, or syntax error, or bad cast

    If you follow the output of make carefully you should be able
    you should be able to see what file the compiler was working
    on.  The error message from the compiler will say what line
    number the error occurred on (or near).  It may also give
    a number for the error, so be sure to guess that you are looking
    at the line number and not the error number.

    Look at the line in the appropriate file.  Don't worry, you won't
    have to change anything in the source code.  This is just to see
    what kind of error it is.

    If on or near that line there is "ANYPTR", then you will should
    edit the predefines in the Makefile to use "-DANYCHARPTR".  Chances
    are that if you have encountered this error you will also encounter
    the more serious errors described next.  Anyway, it won't hurt to make
    the change to the makefile and try again.

    If on or near that line was something like
\begin{verbatim}
blah blah
function(blah blah, blah blah)
{
\end{verbatim}
    It doesn't matter what the `function' or `blah' is, what matters
    is that in the `"("' and `")"' there are (at least) two `blah's between
    any commas or the the parenthesis stuff looks like "(void)".  Also
    this will always be at the begining of a line.

    If that is what you found, it means that your C compiler doesn't
    understand prototypes in function definitions.  If this is what
    happened then, at the moment there is nothing to do, but try a
    more modern C compiler.

    In particular if you are compiling on a Sun workstation.  Do not
    use \prg{cc}, but use \prg{gcc} or \prg{acc}.  There is a tool called
    \prg{unproto} which will fix modern programs
    to work with obsolete C compilers.

    If the syntax error was in the file \file{distrat.h} and near a line that
    looks like
\begin{verbatim}
extern blah function(blah blah)
\end{verbatim}
    then your compiler doesn't even understand prototypes in function
    declarations.  Again, if you have this problem then you probably have
    the above problems as well.  Try another C compiler.
\end{itemize}

\subsubsection{Reporting problems}

Naturally, I would be interested in knowing about problems you
encounter, but my curiosity does not entail a commitment to
fix anything.  But it is certainly more likely that I will be
able to fix things if you give me as complete information as
you can.  Tell me about your system. (on Unix the \cmd{uname -a}
command will tell you.  About your compiler (for gcc the command
\cmd{gcc -v} will tell you).  Most importantly, I need the error
messages.  Please run make putting all make out put into
a file, say \file{make.out}.  On DOS do this with
\begin{alltt}
 make all > make.out
\end{alltt}

On Unix, if you are using \prg{csh} or some derivative of it like \prg{tcsh}
use
\begin{alltt}
  make all >& make.out
\end{alltt}

Use the \& to get the error messages there as well.  If you are using
the Bourne (standard) shell (\prg{sh}) or the things like
it (\prg{bash}, Bourne Again SHell) or the Korn Shell (\prg{ksh})
\begin{alltt}
  make all >make.out  2>&1
\end{alltt}
If you don't know which type of shell you are using, type
\begin{alltt}
 csh
 make all >& make.out
 exit
\end{alltt}
This way, you will use the csh.

\subsubsection{Special DOS Compiling difficulties}\label{sec:dosprobs}

    If you have never used make or a compiler on your machine, it
    is really best for you to use the assistance of someone who
    is familiar with these tools and DOS, since you often need to
    mess around with memory managers to get make and the C compiler
    to run.  (The \dram programs do not need any special
    memory manager to run, it is make, the linker and the compiler
    that need extended or expanded memory or use special memory managers.)

    As an warning, I once attempted making this with exactly the same
    compiler and make on Livia's machine which has sufficient RAM, but
    it turns out that the DOS utility Smartdrv was installed in such a
    way it took too much memory, so make wasn't getting enough to be
    able to it on the the linker.  It took me several hours one day,
    a night's sleep, and an hour the next to identify the problem.  Once
    identified all worked well.  (A DOS expert would probably figure it
    out much faster).

    The novice should feel free to try to compile under DOS, but if
    you get an out-of-memory error during compilation, consult a
    local DOS programer.  And when you start cursing, that cursing
    should be directed not at me, but toward Seattle.  The problems
    aren't my fault nor the fault of the developers of the C compiler
    (unless, of course you are using mcc, but that's just guilt by
    association.)

\chapter[General usage]
  {Using the programs: General issues}\label{ch:general}

Before providing documentation for each of the programs in the
package, there are some general issues which are common to all of the
programs.

Although the command-line syntax of each command is different, there
is a commonality of style and conversions, which, once explained in
one place only, will make the description of individual programs
easier.  So general properties of command line conventions are
explained in section~\ref{sec:commandline}.  If you are very
comfortable with Unix style commands and command lines, you can
probably skip that section.

Many of the programs read or write files which contain causal maps in
a particular format.  The format of the causal map files is then also
described separately from the specific documentation of each program
in section~\ref{sec:mapfile}.

Several programs take lists of map file names.  The format of these
files is discussed in section~\ref{sec:listfile}.

Many of the programs exit with a particular exit states on different
sorts of error conditions.  Many of these are common across these
programs.  These are listed in section~\ref{sec:exitstates}.  If you
don't know what an exit status is, then you don't need the
information in that section.

\section{Modularity and design philosophy} \label{sec:modularity}

Over the past ten years it has become increasingly popular to design
software packages which are large integrated packages that perform
all the required computer tasks for some project.  In such a
system\dash and \dram is not such a system\dash the user calls up an
\scare{environment} on the computer and does everything within that
environment.  \dram, on the otherhand, instead of being a closed
integrated system, is an open modular system.  Modularity is well
worth the price of making the documentation more difficult to write.
With an integrated system you need to know nothing about your
particular computer or how the program interacts with your computer
(ie, you need know nothing about files and directories if that is what
your computer uses), but you need to learn the system of the
integrated package fully.

With a modular set of programs, you need
to learn less about things specific to the programs, but you need to
know a little more about the your computer system (ie, how to create
and view files, etc).  Since every computer system has at least
one\dash and maybe up to a dozen\dash ways of viewing simple
character files there is no need for me to write a file viewer into
\dram, nor is there a need for you to learn a new one.  Modularity
means that the programer does less work and the user has less to
learn about particular programs.  There are other advantages of modularity,
which should become clear below; but the chief drawback is that since
you have the freedom to use, for example, whatever file viewer you
wish to use, I cannot provide full documentation.  This is because while
\emph{you} probably know how to view simple files on \emph{your} computer,
I don't know how to do so on \emph{your} machine.

Modularity also allows the system to grow easily without requiring
central control: Anyone can add a module.  If some user needs some
other form of analysis for maps of the same type as are used here, it
would just be a matter of writing a module that does that and only
that.  Ideally, the new module would be constructed so that it fit in
with the existing modules, but that is up to the creator of the any
new module.

\section{General command line syntax} \label{sec:commandline}

You can skip the rest of this section if you understand fully
this statement:
\begin{quote}
The programs use traditional Unix style options and arguments.
The non-interactive programs write to standard output.  The
program \prg{drsort} is the only program that reads from standard input.
\end{quote}
If you didn't understand that, then read this section and you will by
the end.

\subsection{Command line arguments}

Many of the commands take command line arguments.  For these
programs they are often names of files which contain causal maps.
For example, the program \prg{density} which can be used to calculate
the ratio of arcs in a map to total places there could be arcs.
\begin{alltt}
  density \emph{filename}
\end{alltt}
where `filename' is the name of a file which contains a map.  In this
example `filename' is the command line argument.  Naturally it should
be the name of an actual file on your system.  Some commands take
no command line arguments, some take one, some take more, and some
take a variable number of arguments.

\subsubsection{Command line options} \label{sec:options}
In addition to command line arguments provide on the command line,
most of the programs take a number of options (which are optional
as the name suggests).  These change the behavior of the programs
in certain ways.  Options, sometimes called `flags', are single characters
and are immediately preceeded by a hypen, `\texttt{-}', character.

For example, with this program (and almost all of the others as well)
there is an option \texttt{h} which displays a very terse summary
of the command line syntax for the particular command.  So typing the
command
\begin{verbatim}
  density -h
\end{verbatim}
will display the command line syntax information instead of trying
to calculate the density of anything.  Many of the commands also have
a \texttt{v} option which stands for `verbose'.  When this option is used
to program runs as normal, but may write out additional information about
what it is doing to your computer screen while it is doing it.
\begin{alltt}
  density -v \emph{filename}
\end{alltt}
will calculate the density of the map in the file called `filename', but
will write out additional information as well.  Note that the option
precedes the command line arguments.

Some options take arguments themselves.  In such cases the arguments to
the options are obligatory and must follow the option (but there may
be whitespace between the option and its argument).  The option \texttt{A}
for the density can be given a numerical argument (actually only
0 or 1) which specifies whether arcs can reasonably occur from a node
to itself.\footnote{%
  One would think that an option that only has two possible values would
  be a simple option instead of an argument taking one.  Normally that
  would be the case, but the reason why this option is they way that it
  is will become clear from the discussion of the \prg{distrat} program
  in chapter~\ref{ch:distrat}.}
If the \texttt{A} option is
given the argument `0' then such reflexive arcs (or the lack of them) will
be calculated as part of the density, while if it is set to `1' then
reflexive arcs (or places where they could be) will be ignored.
(See chapter~\ref{ch:density} for more explanation.)  So,
\begin{alltt}
  density -A 0 \emph{filename}
\end{alltt}
will calculate the density of the map in `filename' assuming that arcs
can occur between nodes and themselves.  The command could have
also been typed as
\begin{alltt}
  density -A0 \emph{filename}
\end{alltt}

If we wanted to make the same calculation verbosely, this could be done
with
\begin{alltt}
  density -v -A0 \emph{filenane}
\end{alltt}
or
\begin{alltt}
  density -A0 -v \emph{filename}
\end{alltt}

Options which don't take arguments can be grouped together (with no
space) behind a single hyphen:
\begin{alltt}
  density -hv \emph{filename}
\end{alltt}
although this particular example is a bit silly, since the \texttt{h}
option means that nothing really gets calculated.

Here are some general points to note about options:

\begin{itemize}
\item options are always single characters
\item options, if used, must appear on the command line before
   any command line arguments.
\item option names are case senstive:  `\opt{-A}' is not `\opt{-a}'.
\item if an option takes an argument it, the argument is obligatory
  and the options arguments may or may not be separated from the option
  by space.
\item only options which do not take arguments may be clustered together
 behind a single hyphen.
\item There is no guarantee that an option with a particular name,
  say `x', will have the same meaning for every program.  However,
  when a group of programs are written a part of a package, it
  is usually the case that the program author will keep option
  names as consistant as possible across programs.
\end{itemize}

Many of the programs which read in maps have a special option, \texttt{-f},
which takes as its argument the name of a file which contains a list
of filenames, which in turn contain maps.  Typically when this option
is used the normal command line arguments are no longer needed.  So,
the command
\begin{alltt}
  density -f \emph{listfile}
\end{alltt}
will read in a list of file names from a file named `listfile'.  Note
that the file name `listfile' is the argument to the `f' option, and
is not a command line argument.  It would also be ok to say
\begin{alltt}
  density -f\emph{listfile} -A 0 -v
\end{alltt}

If you were to actually run
\begin{verbatim}
  density -h
\end{verbatim}
the result would be something like
\begin{verbatimtab}
CM Density. Copyright 1993, 1994 by Jeff Goldberg
Usage: density [options] file
   or  density [options] -f listfile
	files contains a map
	listfile contains a list of files
	Options:
		-h	display this message
		-v	set verbose flag [off]
		-d num	set debug flag to num [0]
		-n num	set max. distinct nodes [300]
		-m num	set max. maps in list [200]
		-t num	set type of map file [1]
		-A num	set alpha to num [1]
\end{verbatimtab}

Most of these options haven't been explaned yet, but by looking at
the option that you already know can learn how to read that rather
terse usage message.  The information in the square brackets,
displays the value value which the program is executed with if the
option is not stated.  It should also be noted that not even I
remember which options are available with which programs and what
they are called.  I make heavy use of the `h' option.

\subsection{Saving program output}
With the exception of \prg{askmap}, none of the programs write their
output to a file.  Furthermore there are no options you can specify
which will tell them to do so.  Yet, if there were not some easy way
to save the output of these programs, they would be useless.  Your
command interpreter provides the mechanism that is needed.

With all Unix shells and DOS command interpreters that I know about
one can tell the operating system to \nt{redirect} to screen output
of a program to a file by using on the command line the `\texttt{>}'
symbol followed by the name of the file you want to output to go to.%
\footnote{I know of one shell, \prg{jsh}, which uses Japanese word
order and treats the redirect symbol `\texttt{>}' as a \emph{post}position
instead of a \emph{preposition} so it actually must appear after the
file name and not before.}

Suppose that you wanted to save the output of the \prg{distrat} program
to a file called \file{dr.out}.  Suppose that you had a list of
map files in a file called \file{maplist}, then the command
\begin{alltt}
  distrat -f maplist
\end{alltt}
would display a great deal of information to the screen.  Programs
are said to write to the \nt{standard output} when they write information
to the screen in such a way that the output can be easily redirected
to a file.  The information produced by the \prg{distrat} program for
example, is written to standard output (normally the screen).  If you
use the command line
\begin{alltt}
  distrat -f maplist > dr.out
\end{alltt}
the program still writes to standard output, but your computer
system will treat standard output as the file \file{dr.out}.  If
the file \file{dr.out} existed before this command, it will be overwritten.

Not everything that a program writes to the screen is written to standard
output.  For example, the copyright notice and various error messages are
written to the screen, but are not written to standard output.  This
means that when you redirect standard output from the screen to
a file using \texttt{>} this other information (error messages, copyright
notice) and so on, are still displayed on the screen and not sent to
the redirected standard output file with the other data.

Again, let me remind you that this redirection tool, `\texttt{>}', is
not something written into the program.  Both when you use it and
when you don't the program simply writes data to its standard output.
However, when you don't redirect standard output your computer's
operating system will treat the screen as standard output; and when
you do use output redirection, your computer treats the named file
as standard output.  \prg{distrat} won't even know if it is writting
to the screen or to a file; all it knows is that it is writting to
standard output.

\subsection{Redirecting standard input}

There is a similar notion, \nt{standard input}, which only matters
in the case of one of the programs provided: \prg{drsort}.  \prg{drsort}
expects data from its standard input, which, unless the operating system
is told otherwise, will be the computer keyboard.  It is highly unlikely
that anyone will want to give that program keyboard input; so, in order
to tell the operating system that the standard input of \prg{drsort} should
come from a file, say \prg{dr.out}, the command would be given as
\begin{alltt}
  drsort < dr.out
\end{alltt}
Since \prg{drsort} not only reads from standard input, but also writes
to standard ouput, the output of \prg{drsort} could be redirected
to a file with something like
\begin{alltt}
  drsort < dr.out > \texttt{outfile}
\end{alltt}

There are other things that Unix users can do with standard input
and standard output that introductory material to Unix covers
better than I can here.  In most practical terms, DOS users are limitted
to the sorts of examples shown above.\footnote{%
  DOS command interpreters do provide a pipe simulation, but since the
  operating system itself does not actually handle pipes, the simulation is
  hardly worth the risk of the mess left behind a broken pipe on DOS.}


\section{Map file format} \label{sec:mapfile}

The files (both the files containing maps and lists) can be created
and edited by any ASCII editor such as \prg{Emacs}, \prg{vi},
\prg{Elvis}, \prg{dosedit}, the \prg{Norton editor}, or any other of
scores of such editors.  The files can also be edited using
word-processors such as \prg{Word Perfect} making sure that they are saved
in ASCII or Text form.  In what follows `line feed', `new-line', and
`return' refers to what is sometimes referred by such word processors
as `hard returns' or `hard line-feeds'.
The files can also be created by the \prg{askmap} program which is
the subject of chapter~\ref{ch:askmap}.

The files containing maps must be in a specific form.  In many
respects the form is very strict, while in other respects it is very
flexible.  The best way to understand what the file can be is to look
at the samples (\file{tun01} \file{tun02} \file{tun03})

In some of the examples below, I use the symbol `\verb*| |' to indicate
a single space.
The file \emph{must} begin with the string

\begin{verbatim*}
Map id: 
\end{verbatim*}

The capitalization there matters as does the space between the `p'
and the `i'.  The colon `:' is also required.  This must be followed
by a space and the an identifier.  For example:

\begin{verbatim*}
Map id: Joan-Smith-1
\end{verbatim*}
Here the identifier is `Joan-Smith-1'.  The identifier must contain
no spaces or tabs.  The identifier must be less than 32 characters
long.

The next line of the map file must contain the number of nodes in the
particular map.  So the first few lines of a file may look like:
\begin{verbatim*}
Map id: Joan-Smith-1
12
\end{verbatim*}
The number of nodes in a map must be at least two, otherwise some
of the calculations based on the maps may be undefined.  The
programs are currently written so as to not accept maps that have
fewer than two nodes.  See section~\ref{sec:avmap:empty} for a discussion
of how such maps might arise.

After the first two lines the rules about newlines and spaces become
much more flexible.  Anytime after the second line where a space,
newline, or tab character can be, you can use any number (greater
than zero) of spaces, newlines, tab characters in any combination.

If the number of nodes, as in the example, is 12, then the next 12
numbers must be the numbers of the nodes in the particular map.

\begin{verbatim*}
Map id: Joan-Smith-1
12
35 22 31 45  2 29 17 19 63 53 54  9
\end{verbatim*}

The node numbers can appear in any order, but the order determines
the layout of the association matrix.  That is, the first node number
specified will determine what node the first row and column of the
associate matrix will correspond to.  So for the example we are
working through, the number in row 1 column 4 of the association
matrix indicated the influence of node 35 on node 45.  A number in
row 6 column 2 would indicate the influence of node 29 on node 22.

Following this, the array of influences appears:

\begin{verbatim}
Map id: Joan-Smith-1
12
35 22 31 45  2 29 17 19 63 53 54  9
 0  3  3  1  0  3  3  2  3  3 -1  0
 3  0  3  2  0  2  3  2  0  2  2 -2
 3  2  0  2  0  3  3  2  2  3 -3  2
 2  2  2  0 -2  2  2  0  2  0  0 -1
 2  0  0  0  0  2  0  0  2  0  1  1
 3  2  2  0  2  0  3  3  2  3  2 -1
 3  3  2  0  0  2  0  2  2  0  3  0
 2  3  0  0  0  2  3  0  0  3  0  1
 0  0  2  0  0  2  2  0  0  3  1  1
 3  2  3  0  0  3  0  0  3  0 -1  0
 1 -2  1  0  2  1  1 -3  1 -2  0 -1
 2  3 -1  0  0  2  0  2  2  0  3  0
\end{verbatim}

Again note that anywhere after the line indicating the construct
numbers, all tabs, spaces, and newlines are equivalent.  The same map
could have been written as
\begin{verbatim}
Map id: Joan-Smith-1
12
35 22 31 45 2 29   17 19   63 53 54 9

0 3   3 1 0 3 3  2 3 3 -1 0 3
          0 3 2 0 2 3 2 0 2 2 -2 3 2 0 2 0 3 3 2 2
3 -3 2 2 2 2 0 -2 2

 2 0 2 0 0 -1 2 0 0 0 0 2 0 0 2 0 1 1 3 2 2 0 2 0
3 3 2 3 2 -1 3 3 2 0 0 2 0 2 2 0 3 0 2 3 0 0 0 2 
 3 0 0 3 0 1 0 0 2 0 0 2 2 0 0 3 1 1 3 2 3 0 0 3 0 0 3 0 -1
0 1 -2 1 0 2 1 1 -3 1 -2 0 -1 2 3 -1 0 0 2 0 2 2 0 3 0
\end{verbatim}
although using such a form is not recommended for obvious reasons.
But what users should note here is that the program may not detect
errors in your input while it is reading the file.  This lack of
syntax checking was a deliberate design decision, to make the
reading of the file quicker.  The other way to make a reading
quicker would have been to impose an extremely rigid file form.

Finally, the program allows any material to appear after the map.  So
maps could look like:

\begin{verbatim}
Map id: Joan-Smith-1
12
35 22 31 45  2 29 17 19 63 53 54  9
 0  3  3  1  0  3  3  2  3  3 -1  0
 3  0  3  2  0  2  3  2  0  2  2 -2
 3  2  0  2  0  3  3  2  2  3 -3  2
 2  2  2  0 -2  2  2  0  2  0  0 -1
 2  0  0  0  0  2  0  0  2  0  1  1
 3  2  2  0  2  0  3  3  2  3  2 -1
 3  3  2  0  0  2  0  2  2  0  3  0
 2  3  0  0  0  2  3  0  0  3  0  1
 0  0  2  0  0  2  2  0  0  3  1  1
 3  2  3  0  0  3  0  0  3  0 -1  0
 1 -2  1  0  2  1  1 -3  1 -2  0 -1
 2  3 -1  0  0  2  0  2  2  0  3  0
Nationality: Latvian
Age: 34
Job-Title: Marketing Directory
Node-35: Market share
Node-22: blah blah blah
\end{verbatim}

The \dram set of programs will not make use of that information, but other
programs might.  The \dram programs will not read beyond the last
element of the association matrix.\footnote{%
  The program \prg{askmap} under the rare circumstance that it
  reads a map file does expect a particular bit of extra information.}
Note that if something is
missing from the association matrix (a line, or even a single
element) the program will try to read past the end of the array.
Its behavior then is very hard to predict, but it will most certainly
lead to either erroneous results, or, if you are lucky, an error message.

One more note about the files:  The influences can be floating point
numbers:

\begin{verbatim}
Map id: test-2
4
   3      5      8     12
   0  -1.54      2   2.40
2.00   0.00   -3.2      2
   1  -2.25      0    2.3
   0   -1.1     -2      0
\end{verbatim}

This will allow files produced by the \prg{avmap} program to be usable
by other programs in the package.

Currently, the programs assume that all input maps have at least one
node.  No calculation are performed for empty maps.  However, the
output of \prg{avmap} may be an empty map.

\section{The form of the list files}\label{sec:listfile}

The list file (the file used with the \texttt{-f} option) is simply the name
of each map file.  In the list file, each map file should be on a
complete line by itself and there should be no blank line, nor
extraneous spaces in the file.

A list file might look like:

\begin{verbatim}
hun01.map
hun02.map
hun03.map
\end{verbatim}

\section{Exit states}\label{sec:exitstates}

Many of the programs exit with various exist states depending on
error conditions.  If you don't know what a exit status is or
what to do with one, then you most certainly don't need to
read this section.  Here they are:

\begin{center}
\begin{tabular}{cp{8cm}}
\bfseries Status & \bfseries Meaning\\[5pt]
0 & Normal exit\\
1 & Usage Error\\
2 & File opening error\\
3 & Impossible error (inform author if you get this)\\
4 & Node out of range\\
5 & Memory error (could be out of memory)\\
6 & Error with input data
\end{tabular}
\end{center}



\chapter{The \prg{distrat} program}\label{ch:distrat}

The \prg{distrat} program
calculates the `distance ratio' between causal maps.  For
definitions of the distance ratio (and of causal maps) see
\citeasnoun{MarkoczyGoldberg95:JOM}.
The program is well
suited for the cross comparison a large number of maps and should run
reasonably well on even the machines slowest machines.

\section{Calling \prg{distrat}}

As with many of the other programs that read map files, there are two
forms for calling \prg{distrat}.  One is where two
file name arguments are provided, and each file contains a causal map, eg,
\begin{alltt}
distrat \textit{mapfile} \textit{othermap}
\end{alltt}
In this case \prg{distrat} will calculate the distance ratios between
the maps stored in files \texttt{\itshape mapfile} and
\texttt{\itshape othermap}.  These files are in the form described in
section~\ref{sec:mapfile}.  Of course, as described in
section~\ref{sec:options}, the various options to this command can appear
before the command line arguments.

In the other usage, the `\texttt{-f}' option is provided and
its argument is the file name of a file containing a list of other
file names (see section~\ref{sec:listfile}).
Those other files contain the maps to be compared.  For
example if you wish to calculate the distance ratio between all of
the files \file{map1}, \file{map2}, \file{map3}, \file{map4},
\file{map5}, \file{map6}, \file{map7}, and
\file{map8} you can create a file that just contains the names of the map
files, one file name per line.  Suppose the contents of the file
\file{maplist} is 
\begin{verbatim}
map1
map2
map3
map4
map5
map6
map7
map8
\end{verbatim}
then you can calculate the distances between the maps in all
of the map files with a command like:
\begin{verbatim}
    distrat -f maplist
\end{verbatim}
Of course all of the other options can also be specified, e.g.:
\begin{verbatim}
    distrat -G2 -d 3 -f maplist
\end{verbatim}

\section{Options available with \prg{distrat}}\label{sec:dropts}

In addition to the \opt{-f} option described above, there are a number
of other options.  The options~\opt{-A} through~\opt{-E} can be used
to set the values for the parameters $\alpha$ through $\epsilon$.  The
`factory default values' uses settings $\alpha=1$, $\beta=3$, $\gamma=1$,
$\delta=0$, and $\epsilon=2$, which have the formula reduce to
\possessivecite{LangfieldWirth92} formula~12.  The settings used by
\citeasnoun{MarkoczyGoldberg95:JOM} and \citeasnoun{Markoczy95:thesis} are
the same except that $\gamma=2$.

The options \opt{-d} and \opt{v} can be used to have more information
displayed to the screen while the program is running about, for example,
intermediate calculations.

The options \opt{-m} and \opt{-n} can be used to reset some program
limits on number and size of maps.

The option \opt{-c} can be used to set the cache size, and is discussed
in much more  detail below.

Here is an almost complete list of options.
\begin{optlist}
\item[\texttt{-h}]
        Lists all command line options

\item[\texttt{-d} $N$]
 	This sets the debug level to $N$, the higher the value
	of $N$, the more internal information the program will
	provide about what it is doing.  The default value is
	zero.  (This option may not be supported in future
	versions.)

\item[\texttt{-v}]
	Tells the program to run ``verbosely'' (same as \verb|-d3|)
        (This option may not be supported in future versions.)
  
\item[\texttt{-f} \textit{fname}]
	The file \textit{fname} contains a list of file names all to
	be compared with each other.  If this option is given
	then no files name arguments may be given on the command
	line.

\item[\texttt{-A} $N$]
	Set $\alpha$ to $N$.  Default 1

\item[\texttt{-B} $N$]
	Set $\beta$ to $N$.  Default 3

\item[\texttt{-G} $N$]
	Set $\gamma$ to $N$.  Default 1

\item[\texttt{-D} $N$]
	Set $\delta$ to $N$.  Default 0

\item[\texttt{-E} $N$]
	Set $\epsilon$ to $N$.  Default 2

\item[\texttt{-n} $N$]
	Sets the maximum node number to $N$.  All node numbers
	in maps must be between 1 and $N$ inclusively.  By
	default $N$ is 300.
	See the discussion on optimizing performance
	in section~\ref{sec:dr:optomize}
        for more about this option.

\item[\texttt{-m} $N$]
	Sets the maximum number of maps to $N$.  This is number
	maximum number of names in the listfile.
	Default $N$ is 200.
	This option is only relevant in conjunction with \texttt{-f}.
	See the discussion on optimizing performance
	in section~\ref{sec:dr:optomize}
        for more about this option.

\item[\texttt{-c} $N$]
	Sets the maximum number of maps to be remembered
	when comparing maps from a list.  If there are more
	than $N$ maps in the list, maps will sometimes have
	to be ``forgotten'' and then re-read from the file.
	The default size of the map cache ($N$) is 40.
	This option is only relevant in conjunction with \texttt{-f}.
	See the discussion on optimizing performance
	in section~\ref{sec:dr:optomize}
        for more about this option.

\item[\texttt{-t} $N$]
	Sets the format type of the map files to be read in.
	$N$ must be a number in the range 1--9.  Currently
	only type 1 is available, and is discussed in 
	section~\ref{sec:mapfile}.  The program
	has been written so that it is easy for other programers
	to add new file formats.  The notes in the source
	file \file{readmap.c} should be consulted for further details.
	Generally, numbers 7, 8, 9 will be used for file
	formats specific to your site.  Lower numbers, such as
	3, 4, 5 and 6 should be used for formats in more
	general use.

	Default is 1.
\end{optlist}

\subsection{The basic options}

The use of \opt{-f} has already been described.  This option is
probably the most common one.  Also, option \opt{-h} has been described
in section~\ref{sec:options}.  The options \opt{-v} and \opt{-d} can
be used by the user to get some information about the internal state
of the program during its operation.  However, exactly what information
is provide for different settings of \opt{-d} or for using \opt{-v}
is subject to a tremendous amount of variation from one version of
the program to the other.

\subsection{Parameters}

The meanings of the options for setting $\alpha$ through $\epsilon$ are
explained in the paper by \citeasnoun{MarkoczyGoldberg95:JOM} and not
in the documentation for the software.  The same is true ofthe
actual meaning of the distance ratio calculation.

\subsection{Map limit options}
The options \opt{-n} and \opt{-m} only need to be reset from the default
if you are reading more maps than the default value for \opt{-n},
or your highest node number for a map is greater than the default
value for \opt{-m}.

\subsection{Map cache}
The option \opt{-c} can be adjusted when you are reading a large
number of maps (or very large maps).  The value of the argument of
this option is the maximum number of maps which remain in the program's
memory (the map cache) when the program is running.  It is only relevant
when the \opt{-f} option is used.  If the value of \opt{-c} is
greater then the actual number of files to be read in, then the
size of the map cache is automatically set to the lower amount.
Generally, if a map remains in the map cache, then it will not
have to be reread from the disk file.

\subsection{Optimaizing speed and memory}\label{sec:dr:optomize}
You only need to be read this section if you find yourself running
into either memory limitations, or need to get the program to run
faster.

By far, the most substantial savings in both time and memory can be
gained by adjusting the cache size with option~\opt{-c}.  If, for example,
you will be calculating the distance ratios between 120 maps, you can
acheive a much faster run by increasing the size of the map cache.  If
the maps are large (say 100 actual\footnote{%
  That is the actual number of nodes in a map, as distinct from the
  question of the highest possible node number, is what matters here.}
 nodes each) then you may well find that
even the default map cache size leads to memory problems on some limited
machines.\footnote{%
  The distributed versions for DOS can only make use of native DOS
  memory (something I consider a design error of DOS and not of the
  programs).  However, if you have a 32 bit DOS machine, then you can
  obtain versions of the program that will make use of all free memory
  on your machine.  If you are running on a machine with a real operating
  system, (ie, one which gives programs the memory they ask for if
  the memory is available without asking the programs to jump through
  a few hoops first) then you are less likely to run into memory
  problems.}
To date, I have not heard reports of users needing to adjust the default
for their runs, though I do know of one case where running on
111 maps (of 10 nodes each), increasing the cache size to 70 made
a dramatic improvement in speed.

Adjusting the options \opt{-m} and \opt{-n} downward to reflect the
actual size will have an extremely small improvement on both time and
memory, but this really will be so small as to not be worth it.  A small,
but far greater improvment in memory can be achived by keeping the file
names in the list file short.  And those files will be read from the
disk much faster if they are in the same directory that \prg{distrat}
is called from.

Finally, I would be interested to hear of any cases in which users
have felt the effect of machines limits running these programs.  Writing
a program to perform the calculation was trivial, having it do so
making good use of the computer's resources was the interesting (and
fun) bit.

\section{The output}\label{sec:drout}

\prg{distrat} writes its output to the screen (standard output) and
might look like
\begin{verbatim}
DR(hun01, hun02) = 0.1234
DR(hun01, hun03) = 0.5678
DR(hun02, hun03) = 0.9012
\end{verbatim}

\subsection{Order of the output information}

Note that the the ordering of distance ratios will be an upper right
corner of a distance matrix.  So if comparing maps 1 through 6 (as
the order in the list file) then the $N$th distance ratio will fit
according to the following table

\begin{center}
\begin{tabular}{l|rrrrrr}
  &   1 &   2 &   3 &    4 &    5 &    6\\
\hline
1 &     & 1st & 2nd &  3rd &  4th & 5th\\
2 &     &     & 6th &  7th &  8th & 9th\\
3 &     &     &     & 10th & 11th & 12th\\
4 &     &     &     &      & 13th & 14th\\		
5 &     &     &     &      &      & 15th\\
6 &     &     &     &      &      &
\end{tabular}
\end{center}

A more general statement of the output order is
\[
\begin{array}{cccccc}
(1, 2)& (1, 3) & (1, 4) & \ldots & (1,n-1)    &  (1, n)\\
      & (2, 3) & (2, 4) & \ldots & (2,n-1)    &  (2, n)\\
      &        & (3, 4) & \ldots & (3,n-1)    &  (3, n)\\
      &        &        & \ddots & \vdots     & \vdots\\
      &        &        &        & (n-2, n-1) & (n-2, n)\\
      &        &        &        &            & (n-1, n)
\end{array}
\]
That is, it does the first with the second, then the first with the third,
first with the fourth and so on until it does the first with the last, then
it does the second with the third, the second with the forth and so on
until it does the second with the last, then the third with the forth, and
so on.

The notion of `first', `second' and so on has to do with the order
in which the files are listed in the \textit{\tt listfile}.

\subsubsection{Getting the ouput in order}\label{sec:sortdrout}

Most software packages which input distance data seem to prefer input
as lower left instead of upper right.  Also, they like to have the
zero's down the diagonal.  
If you wish to sort the output in this way, there are a number
of things you can do.

\begin{itemize}
\item One way is to use the \prg{drsort} program which is discussed
  in more detail in chapter~\ref{ch:drsort}.   It has been designed
  exactly for this purpose.

\item Another is to use existing sorting tools.  This requires some planning
  ahead, in that it means that the order of the maps in list file
  should match the sorting order of the map id's.  That is, if the map
  in file xxx has the id Jim-Jones and the map in file yyy has the id
  John-Smith, then you can convert the file from an upper right
  triangle by using your sorting program to sort the lines in your
  output file with the primary sort on the second field and the
  secondary sort of the first field.  If your sorting software can't do
  this, it is time to get a new sorting program.

\end{itemize}

\chapter{The \prg{askmap} program} \label{ch:askmap}

The \prg{askmap} program is probably the most difficult to document, and
the documentation that you see for it here is not as complete as the
documentation for the other programs.  However, \prg{askmap} is an
interactive program which prompts the user for input and will do its
job as required.  The \prg{askmap} program is inflexible in a number
of ways, but for normal use it can be extremely helpful.  It is the program
what will should be changing the most radically from release to release.

It must be reiterated here that these programs along with this
documentation do not substitute in any way an understanding of
the relevant sections of \citeasnoun{MarkoczyGoldberg95:JOM}.
The \prg{askmap} program assists with the elicitation of the
causal map from the subject, after the subject has selected
her/his top constructs.\footnote{%
 If you don't know what this means, then stop now and read
 the paper.}
\emph{The program is not designed to for self-administration!}  It is
designed to save the researcher the trouble finding the next pair
of questions and of writing down the answer.

\emph{The program does no automatic back-up of data entered.  Always use
audio recording (with the subject's permission) of the interview
to recover data in case of power failure, computer problem or problem
with \prg{askmap}.}

\section{What it does}

\prg{Askmap} will prompt the interviewer for a file name for the
created map to be stored in, for the "Map id" (see section~\ref{sec:mapfile}),
for the number of constructs (nodes) in the map, and then for the
number of each node, and, optionally, a brief text for each node.

Once it has all of that information, it will then prompt for pairs of
influencing relations, which should be entered as integers in the
range of possible values\footnote{%
  The current version of the program prompts for input in the range
  $[-3,+3]$ although it will accept any small integer value.  Something
  may be done to make the this prompt more flexible in future versions,
  but at the moment the moment ignore the values in the prompt if you
  wish to use another range of numbers.}.
It uses so-called Ross ordering to determine which pair of pairs should
be asked next.  When it is done, it will write out a file in the form
described in section~\ref{sec:mapfile}.  That file can be used as
a data file for other programs (such as \prg{distrat}) without any
other processing of the file being needed.

The best way to see who the program works is to give it a small try
now.  First select a small number of constructs, say 5, from some
list or other, and then simply call up the program,
\begin{alltt}
  askmap
\end{alltt}
and follow the instructions.

\section{Correcting input errors}

If something is input by error, or if the subject changes an opinion, there
is nothing you can do with the program after the information has been
entered.  The program is \emph{not} a cause map editor.  In most
cases you can note the error on a sheet of paper (have it ready before you
start \prg{askmap}) and correct it by editing the created file after all is
done.  Even with \prg{askmap} you will not get away with failing to
read section~\ref{sec:mapfile}.

\section{Saving work in progress and quiting}

When \prg{askmap} prompts for an influence
between a pair of nodes $A$ and $B$ it will ask a pair
of questions.  The first will be the influence of $A$ on $B$ and the
second will be an influence of $B$ on $A$.   When it asks for the
first pair, you may, at your option enter the the letter "s".
\begin{alltt}
...
4th pair 30% done
"three" ---> "two"
Influence of node 3 on 2 [-3 to 3]: s
Saving map in file askmap.tmp
4th pair 30% done
...
\end{alltt}
which saves the work so far in a file called \file{askmap.tmp}.  If you
have a file already by that name, it will be overwritten; so you should
never start \prg{askmap} if you have important information only
in that file.

It is highly recommended that at any pause or interruption
during the interview you do use the "s" command.

\subsection{Quitting}
At exactly the same time as the "s" command can be used, the "q" command,
for ``quit'' can also be used.  This will first ask you whether you
want to same the file, and then it will quit.  Note that here too, the
information will be saved to a file called \file{askmap.tmp}.  You should
immediately copy that file to a different name.

At the moment the program does not allow you to save or quit at any other
time.  It is only at the first influence of a pair of nodes.

If you need to quit after the first of a pair of questions has been
answered, then provide a fake answer to the second of the pair, and quit
at the begining of the next pair.  You will have to make a change to the
file \file{askmap.tmp} described below.

\subsection{Restarting with a partial map}

If you have saved a map in \file{askmap.tmp} and copied it to,
say \file{saved.tmp} you can restart askmap with
\begin{alltt}
askmap -r saved.map
\end{alltt}
\prg{Askmap} will ask you, again, for the name of the file where the
completed map is to be saved, and it will ask again for the optional
text that goes with each node since none of this information is actually
saved in the created map or \file{askmap.tmp}.  It will then restart
asking pairs from the point at which you last saved.

It knows where to restart because the last line of \file{askmap.tmp}
has a line that looks like
\begin{alltt}
AM-Recover-from-pair: \textit{N}
\end{alltt}
where $N$ is the $N^{\rm th}$ pair to in the Ross ordering, and should
be the last pair actually completed.  So, if you save and quit on the
eigth pair, $N$ will be~7.

If you had to add a fake answer in order to quit, you should
edit the file with the partial map in it to reduce the point
of recovery by one.  So, if you have completed the first half of
pair 10, and had to provide a fake answer to get to the next pair so
that you could save and quit; then you should edit \file{askmap.tmp}
(or whatever name you copied it to) so that the last line will
have \texttt{9} instead of \texttt{10} as the pair to recover
from.\footnote{It would probably have been easier for me to fix the program
 than to document this.}

\section{Using a different language}\label{sec:amlfiles}

The messages which askmap writes to the user can appear in any language
in addition to English for which there is a
\texttt{\textit{language}.aml} file, where \texttt{\itshape language}
is the name of the language.  With this distribution, the files
\file{english.aml} and \file{magyar.aml} (Hungarian) are included.
The file \file{english.aml} is
never needed since English responses are built into the
program, but is provided as a basis for translation if you wish
to create your own \file{.aml} file.

One would call \prg{askmap} to run in Hungarian with the following
command:

\begin{alltt}
askmap -L magyar
\end{alltt}

In the current version, the file \file{magyar.aml} must be copied
to the directory you call \prg{askmap} from.
This is an unfortunate limitation,
but it will remain until some further release.

\subsection{How to create your own \file{.aml} file}

Recall that this program is not intended for self-administration
of the causal map elicitation technique; but it is still useful to
have the prompts from the program in the same language that you
are using with the subject.  Creating an \file{.aml} file for
use with \prg{askmap} will require a fair bit of technical expertise,
and\dash as with all translations\dash more time than you might expect.


\begin{center}
\itshape [This section not written yet.  Contact the authoer if you
wish to create your own \file{.aml} file.]
\end{center}

\section{Limits and things to improve}

\subsection{Limits}

\begin{itemize}
\item{} \prg{Askmap} does no checkpointing or intermediate back-ups.
    A power failure, program bug or whatever could very easily lead
    to the loss of data.  Always audio record data elicitation sessions
    as back-up.

\item{} \prg{Askmap} can query maps with no more than 181 elements due to
    limits in the Ross ordering routine.  See chapter~\ref{ch:ross} for
    more information about this.

\item Most strings (map id, texts which go with nodes, etc) are limited
   in length to 40 characters each.
\end{itemize}

\subsection{Things to improve}

This is a list of things that I know should be improved and may
be in future versions (but I need to hear from users exactly how
annoying the current limits are).

\noindent
Regarding basic operations:
\begin{itemize}
\item
 More error checking of responses to make sure that
   reasonable responses are given.  (Currently, sanity checks on input
   are sporadic.)
\item
 The Ross ordering routine is wasteful of memory because
   I don't have a nice definition of the appropriate function in
   a computer friendly way.  It should be rewritten entirely.
\item
 Signal processing (so things like Ctrl-C would work) should be
   added.
\item
 Add option for suppressing page clears between querying pairs.
\end{itemize}

\noindent
There are things to improve regarding language facilities.
\begin{itemize}
\item
 Set things up so that the \file{.aml} files do not have to be in the
   current directory.
\item
 Allow environment variable for language.
\item
 Switch from octal coding of extended ASCII to decimal.
\item
 Character-set dependence/assumptions stated in \file{.aml} files instead
   of being built in.
\item
 Solution to identical first letter for ``yes'' and ``no'' in some language
   should be found.
\end{itemize}

\noindent
Things need to be done regarding the way partial files are saved
and \prg{askmap} is restarted.
\begin{itemize}
\item Allow saves (and quits) half way through pairs.
\item Save (and restore) node text information.
\item Check file overwrite for \file{askmap.tmp}
\item Allow user specified save file.
\end{itemize}

\chapter{The \prg{drsort} program}\label{ch:drsort}

The \prg{drsort} program is designed to transform the output of the
\prg{distrat} program discussed in chapter~\ref{ch:distrat},
so that that output could be used as input to
a variety of programs for further (usually statistical) analysis.

The \prg{distrat} program produces its output one distance ratio per line.
The details are discussed in section~\ref{sec:drout} which should
be reviewed before proceeding with this chapter.  That is, the
problem that \prg{drsort} was create to solve is described there
and will not be repeated here.

For the example that follow, we will work with a rather contrived
output file from \prg{distrat}.  This~\prg{distrat} output is
listed in figure~\ref{fig:droutput}.
Note that this sample file is contrived so that the distance ratio
between test-$N$ and test-$M$ bares a striking relation to $N$ and $M$.

\begin{figure}
\newcommand{\omitted}[1]{\vspace{.5ex}%
   {\normalfont\itshape [\ldots #1 lines omitted\ldots]}
   \vspace{.5ex}}
\caption{Contrived \prg{distrat} output and \prg{distrat} input}%
   \label{fig:droutput}
\begin{alltt}
DR(test-01, test-02) = 0.0102
DR(test-01, test-03) = 0.0103
DR(test-01, test-04) = 0.0104
\omitted4
DR(test-01, test-09) = 0.0109
DR(test-01, test-10) = 0.0110
DR(test-02, test-03) = 0.0203
DR(test-02, test-04) = 0.0204
\omitted5
DR(test-02, test-10) = 0.0210
DR(test-03, test-04) = 0.0304
DR(test-03, test-05) = 0.0305
\omitted{23}
DR(test-08, test-09) = 0.0809
DR(test-08, test-10) = 0.0810
DR(test-09, test-10) = 0.0910
\end{alltt}
\end{figure}

The \opt{-h} option will provide a brief usage message about the command
line options

\begin{verbatimtab}
drsort: Copyright Jeff Goldberg 1994
Usage:
	drsort [-lufezh] [-d N]
		-h	this message
		-l	Lower left corner [ON]
		-u	Upper right corner [OFF]
		-f	Full matrix (implies -z) [OFF]
		-z	Zero's in diagonal [ON]
		-e	Empty diagonals [OFF]
		-n	Print variable names [OFF]
		-c	Print number of variables [OFF]
		-d N	Debug info level [0]
\end{verbatimtab}

As you can see there are many option, but you won't need most of them.

The program reads from standard input (You should review
section~\ref{sec:commandline} if you don't understand this or what
the `\verb|<|' and `\verb|>|' symbols mean in the following
examples.)  For example if we take figure~\ref{fig:droutput} to be
the contents of the file \file{shortdr.out} then a command like
\begin{alltt}
  drsort < shortdr.out
\end{alltt}
will write to standard output:
\begin{alltt}\small
 0.0000
 0.0102 0.0000
 0.0103 0.0203 0.0000
 0.0104 0.0204 0.0304 0.0000
 0.0105 0.0205 0.0305 0.0405 0.0000
 0.0106 0.0206 0.0306 0.0406 0.0506 0.0000
 0.0107 0.0207 0.0307 0.0407 0.0507 0.0607 0.0000
 0.0108 0.0208 0.0308 0.0408 0.0508 0.0608 0.0708 0.0000
 0.0109 0.0209 0.0309 0.0409 0.0509 0.0609 0.0709 0.0809 0.0000
 0.0110 0.0210 0.0310 0.0410 0.0510 0.0610 0.0710 0.0810 0.0910 0.0000
\end{alltt}

As you can see, by default the output is a lower left corner with
zeros in the diagonal.  If you wanted to save this output, let's
say into a file called \file{statdist.dat} you could do something like:
\begin{alltt}
  drsort < shortdr.out > statdist.dat
\end{alltt}
And now the file statdist.dat will contain the output which
you can examine with with \cmd{cat} or \cmd{more} (or \cmd{TYPE}
on DOS).

On systems where using pipes makes sense you can pipe the
out put of \prg{distrat} directly to \prg{drsort}
\begin{alltt}
 distrat -f list | drsort
\end{alltt}
which would perform the distance ratio calculation and present it neatly
arranged to standard output.  Of course that output could also be redirected

\begin{alltt}
  distrat -f list | drsort > statin.dat
\end{alltt}
And now \file{statin.dat} contains the matrix.

The program will only accept input which is in the form of the
output of the \prg{distrat} program.

\section{Different output forms}

By default, \prg{drsort} gives a lower left triangle with zeros in the
diagonal and no other information.  If you wish to change that
you need to use one of the options.

\subsection{Upper right corner: option \opt{-u}}

If you want an upper right triangle, use the \opt{-u} option

\begin{alltt}
 drsort -u < shortdr.out
\end{alltt}
yields
\begin{alltt}\small
 0.0000 0.0102 0.0103 0.0104 0.0105 0.0106 0.0107 0.0108 0.0109 0.0110
        0.0000 0.0203 0.0204 0.0205 0.0206 0.0207 0.0208 0.0209 0.0210
               0.0000 0.0304 0.0305 0.0306 0.0307 0.0308 0.0309 0.0310
                      0.0000 0.0405 0.0406 0.0407 0.0408 0.0409 0.0410
                             0.0000 0.0506 0.0507 0.0508 0.0509 0.0510
                                    0.0000 0.0607 0.0608 0.0609 0.0610
                                           0.0000 0.0708 0.0709 0.0710
                                                  0.0000 0.0809 0.0810
                                                         0.0000 0.0910
                                                                0.0000
\end{alltt}


\subsection{Full matrix: option \opt{-f}}

If you want a full symmetrical matrix, use the -f option
\begin{alltt}
  drsort -f < shortdr.out
\end{alltt}
produces on the standard output:
\begin{alltt}\small
 0.0000 0.0102 0.0103 0.0104 0.0105 0.0106 0.0107 0.0108 0.0109 0.0110
 0.0102 0.0000 0.0203 0.0204 0.0205 0.0206 0.0207 0.0208 0.0209 0.0210
 0.0103 0.0203 0.0000 0.0304 0.0305 0.0306 0.0307 0.0308 0.0309 0.0310
 0.0104 0.0204 0.0304 0.0000 0.0405 0.0406 0.0407 0.0408 0.0409 0.0410
 0.0105 0.0205 0.0305 0.0405 0.0000 0.0506 0.0507 0.0508 0.0509 0.0510
 0.0106 0.0206 0.0306 0.0406 0.0506 0.0000 0.0607 0.0608 0.0609 0.0610
 0.0107 0.0207 0.0307 0.0407 0.0507 0.0607 0.0000 0.0708 0.0709 0.0710
 0.0108 0.0208 0.0308 0.0408 0.0508 0.0608 0.0708 0.0000 0.0809 0.0810
 0.0109 0.0209 0.0309 0.0409 0.0509 0.0609 0.0709 0.0809 0.0000 0.0910
 0.0110 0.0210 0.0310 0.0410 0.0510 0.0610 0.0710 0.0810 0.0910 0.0000
\end{alltt}

\subsection{Empty diagonals: option \opt{-e}}

If you want the diagonal to be empty, use the \opt{-e} option
\begin{alltt}
 drsort -e < shortdr.out
\end{alltt}
produces
\begin{alltt}\small
 0.0102
 0.0103 0.0203
 0.0104 0.0204 0.0304
 0.0105 0.0205 0.0305 0.0405
 0.0106 0.0206 0.0306 0.0406 0.0506
 0.0107 0.0207 0.0307 0.0407 0.0507 0.0607
 0.0108 0.0208 0.0308 0.0408 0.0508 0.0608 0.0708
 0.0109 0.0209 0.0309 0.0409 0.0509 0.0609 0.0709 0.0809
 0.0110 0.0210 0.0310 0.0410 0.0510 0.0610 0.0710 0.0810 0.0910
\end{alltt}
The \opt{-e} and the \opt{-u} option can be used together to get an
upper right corner with empty diagonals.

\subsection{Map names: option \opt{-n}}

The option \opt{-n} provides the map ids across the top:
\begin{alltt}
  drsort -n < shortdr.out
\end{alltt}
produces something like (with bits left out)
\begin{alltt}\small
test-01 test-02 test-03 test-04 test-05 test-06 test-07\ldots
 0.0000
 0.0102 0.0000
 0.0103 0.0203 0.0000
 0.0104 0.0204 0.0304 0.0000
 \ldots
\end{alltt}

\subsection{Map count: option \opt{-c}}

If you wish to have the number of maps also produced, then use
the \opt{-c} option.
\begin{alltt}\small
10
 0.0000
 0.0102 0.0000
 0.0103 0.0203 0.0000
 0.0104 0.0204 0.0304 0.0000
 \ldots
\end{alltt}
It is particularly useful in combination with the \opt{-n} option.
When they are both used, the count (number of maps) is always
written before the names of the maps.

\subsection{Conflicting options}

If you try to use conflicting options (\opt{-f} and \opt{-u} for example) the
result will be unpredictable.  Actually the results are fully
predictable; it does the same thing each time.  What ``unpredictable''
really means in software documentation is that the result might
change in future releases, so you shouldn't count on any particular
behavior here.


\section{Time and Space}

If you are sorting a large number of ratios, \prg{drsort} may
run out of space.  Here's why:

As someone pointed out, time and space are the Yin and Yang of
programming.  This program tries to find its way through this
in a reasonable manner.  The amount of space for floating
point numbers (floating point units, $F$)
reserved by the program at its point of maximal memory usage is:
$F = p + m - 1$
where $m$ is the number of maps, and $p$ is the number of
distance ratios between maps (ie, $p= (m^2-m)/2$); so
$F$ works out to be $(m^2+m)/2 -1$.

The number of memory pointers, $P$, at this maximum time will
actually be nearly the same: $P = p + m$.

So the total maximum memory usage $M$ of the program will be
\[
M = S_r + (S_f + S_p)\times(m^2+m)/2)
\]
where $S_r$ is the actual size of the program, $S_f$ is the size
of a floating point unit, and $S_p$ is the size of a memory pointer
on the machine.
A typical size of these things on many machines would be for 
$S_f$ to equal 4 bytes, and $S_p$ equal 4 bytes and for the size
of the program, $S_r$, to be about 30 thousand bytes.
point number to require 4 bytes and for each pointer to require 4
bytes and for the program to be about 30 thousand bytes.  So, under
such a configuration memory usage is
\[
M = 30000 + 4(m^2+m)
\]
in bytes.

So for 100 maps about 70K of memory is needed.  While for 350 maps
about 522K of memory would be needed.  This strikes me as a
sufficiently large number, since on systems with restricted memory
you would probably not be able to get your statistical package to
work on that much data anyway.  Yet you should be aware that the memory
requirements go up with the sqaure of the number of maps.
If you do need to use \prg{drsort} on output from large numbers
of maps, then do get in touch with the author who can provide
you with versions of the program that will use all available memory
on your machine.

It would have been possible to write the program to use less memory,
but would have either required the user to tell the program
how many maps were used or have the program make two passes
over the input (the first time through to count its length).

\section{Notes}

This program will not be built in the the \prg{distrat} program
since it would force \prg{distrat} to store all its
output.  The use of the pipe (\cmd{distrat \textit{args} | drsort} ) should
be sufficient.

\chapter{The \prg{density} program}\label{ch:density}

The density of a map is the ratio of arcs to places where there could
be arcs.  For example, in a 5 node map in which we do not consider
reflexive arcs (arcs from a node to the same node) there are 10
places were there could be arcs.  If there are ten arcs within such a
map, then the map has a density of 1.0, if there are none, then the
density is 0.0.  If the map has, say, six arcs then the density is
$6/10$, or 0.6.  Density is not used or discussed by \MG, but the program
was trivial to write once the others were written.

Usage and input files are as will other programs in the
package.  So run it with the \opt{-h} option of a list of
command line options:

\begin{verbatimtab}
Usage: C:\BIN\DENSITY.EXE [options] file
   or  C:\BIN\DENSITY.EXE [options] -f listfile
	files contains a map
	listfile contains a list of files
	Options:
		-h	display this message
		-v	set verbose flag [off]
		-d num	set debug flag to num [0]
		-n num	set max. distinct nodes [300]
		-m num	set max. maps in list [200]
		-t num	set type of map file [1]
		-A num	set alpha to num [1]
\end{verbatimtab}

All of the command line options here have exactly the same meaning
that they have for the \prg{distrat} program.
So review the options described in \fullref{sec:dropts} for
a complete description.

The output of the program is fairly straight forward.
\begin{alltt}
  density tun01
\end{alltt}
produces
\begin{alltt}
Density(TUN-0792-01) = 0.7333
\end{alltt}


\chapter{The \prg{iodeg} program}\label{ch:iodeg}

The \prg{iodeg} program is used to calculate the \nt{in degrees} and
\nt{out degrees} for various nodes in a map.

These concepts are defined and used in several cause mapping papers 
\citeaffixed{BougonWeickBinkhorst77,FordHegarty84}{e.g.,}.  In
addition to the standard senses of these terms, the \prg{iodeg}
calculates not only the number of arcs to each node, but also
sums up the absolute values of their strengths.   This additional
type of information was used by \citeasnoun{FordHegarty84} and
by \citeasnoun{Markoczy95:thesis}.  In and out degree calculations
are not described in \MG.

\begin{figure}
\caption{Map in file \file{tun01.map}}\label{fig:tun01map}
\begin{alltt}
Map id: TUN-0792-01
10
 2  6 13 27 36 37 41 43 47 49

 0  2  0  0  3  3  1  1  2  0
 0  0  3  3  3  3  3  2  3  3
 3  3  0  0  1  3  3  3  1  0
 2  2  0  0  2  3  2  0  2  0
 3  3  2  3  0  3  1  0  0  0
 3  3  0  3  2  0  3  0  0  2
 3  2  3  0  3  3  0  0  3  2
 0  2  3  0  2  3  3  0  3  0
 1  3  3  2  3  3  2  0  0  0
 2  3  1  2  3  2  0  0  2  0
\end{alltt}
\end{figure}

For example, if the map \file{tun01.map} looks like the map in
figure~\ref{fig:tun01map} then the output of \cmd{iodeg tun01.map}
will look like:

\begin{alltt}
In-degrees for map TUN-0792-01:
		    2:     7  (+17.000)
		    6:     9  (+23.000)
		   13:     6  (+15.000)
		   27:     5  (+13.000)
		   36:     9  (+22.000)
		   37:     9  (+26.000)
		   41:     8  (+18.000)
		   43:     3  (+6.000)
		   47:     7  (+16.000)
		   49:     3  (+7.000)
Out-degrees for map TUN-0792-01
		    2:     6  (+12.000)
		    6:     8  (+23.000)
		   13:     7  (+17.000)
		   27:     6  (+13.000)
		   36:     6  (+15.000)
		   37:     6  (+16.000)
		   41:     7  (+19.000)
		   43:     6  (+16.000)
		   47:     7  (+17.000)
		   49:     7  (+15.000)
\end{alltt}
The first column is the node number, the second is the in-degree
or out-degree of the node (number of arcs leading to it, number
of arcs leading from it).  The third number is the sum of the
absolute values of the weights of the arcs leading in (or out).

The options for the program can be found by using the \opt{-h}
option, which will tell you the following.
\begin{verbatimtab}
Usage: iodeg [options] file
   or  iodeg [options] -f listfile
	files contains a map
	listfile contains a list of files
	Options:
		-h	display this message
		-v	set verbose flag [off]
		-d num	set debug flag to num [0]
		-n num	set max. distinct nodes [300]
		-m num	set max. maps in list [200]
		-t num	set type of map file [1]
		-A num	set alpha to num [1]
\end{verbatimtab}
The meaning of all of the options here are identical to the options
for the \prg{distrat} program described fully
in section~\fullref{sec:dropts}.

\prg{Iodeg} does not sort the output.  If you want it sorted
or transformed in some systematic way, that is up to you.  Any
good text editor will be able to perform most transformations as
will programs designed for such purposes such as \prg{perl},
\prg{awk}, and \prg{sed}, which are standardly available on
Unix systems and are also freely available for DOS from
standard internet archives.



\chapter{The \prg{avmap} program}\label{ch:avmap}

The \prg{avmap} program performs the calculation described in
\MG~[pp.~320--321] as the way of creating so-called \nt{central maps}.

To get the average (central) map for the three maps \file{tun01},
\file{tun02} and \file{tun03} use the command:
\begin{alltt}
  avmap tun01 tun02 tun03
\end{alltt}
which should result in
\begin{alltt}
Map id: Average-map
7
     8     27     36     37     40     41     47

 0.000 -2.000 -1.000  3.000 -3.000  0.500  1.500
 1.000  0.000  2.000  3.000  0.000  1.500  2.500
 0.000  3.000  0.000  3.000  0.000  0.500  0.500
 1.000  3.000  2.000  0.000 -1.000  3.000  1.500
-2.500  0.000  1.000  3.000  0.000  0.500  2.000
-0.500  1.000  1.000  3.000  1.000  0.000  3.000
-0.500  2.000  2.000  3.000  0.500  2.333  0.000
\end{alltt}

As with \prg{distrat} and most of the programs here, the \opt{-f} option
can be used so that the file names come from a list.
So \cmd{avmap -f list} yields,
\begin{alltt}
Map id: Average-map
3
    36     41     47

 0.000  0.143  1.200
 0.000  0.000  3.000
 2.200  2.600  0.000
\end{alltt}

You can use normal command line redirection
(see section~\ref{sec:commandline} begining on page~\pageref{sec:commandline})
to save the results in a file:
\begin{alltt}
 avmap tun01 tun02 tun03 > tun123.avr
\end{alltt}
places the output map into the file \file{tun123.avr}, which can
then be used as a map file as input for other programs.

For the usual usage message use the \opt{-h} option:
\begin{verbatimtab}
Usage: C:\BIN\AVMAP.EXE [options] file [files]
   or  C:\BIN\AVMAP.EXE [options] -f listfile
	file and files contain maps
	listfile contains a list of files
	Options:
		-h	display this message
		-v	set verbose flag [off]
		-d N	set debug level flag [0]
		-p num	set cutoff percent to num [50]
		-n name	set map id to name [Average-map]
\end{verbatimtab}

\section{The options}

\subsection{Changing the output map-id}

One of the most important options are the \opt{-n} option which
allows you to give the created map an id distinct
from ``Average-map''.  For example
\begin{alltt}
 avmap -n TUN-averages -f list
\end{alltt}
yields
\begin{alltt}
Map id: TUN-averages
3
    36     41     47

 0.000  0.143  1.200
 0.000  0.000  3.000
 2.200  2.600  0.000
\end{alltt}
Note that the here the map id of the output corresponds to what
was specified as the argument to the \opt{-n} option on the
command line.

Naturally you should make the name something which is a legal
string for a map id (see the map file format section,
section~\fullref{sec:mapfile}).
Also, if you use characters that may mean something
special to your command interperter, then you will need to quote
the name or otherwise escape the special meanings.  I can't tell
you exactly what these characters are or what to do because
that depends on your command interperter.

\subsection{Changing the cut-off percentage}

The other important option is \opt{-p} which sets the minimum percentage of
maps that a node must be in to appear in average.  The default
is 50.  By using a lower percentation we can get larger maps.
\begin{alltt}
 avmap -p40 -flist
\end{alltt}
produces
\begin{alltt}
Map id: Average-map
4
    27     36     41     47

 0.000  1.800  1.000  1.750
 2.600  0.000  0.143  1.200
 0.750  0.000  0.000  3.000
 1.500  2.200  2.600  0.000
\end{alltt}

\subsection{The \opt{-v} and \opt{-d} options}

The options \opt{-v} and \opt{-d} are there to provide
you with debugging information about intermediate calculations.

\section{Empty and single node maps}\label{sec:avmap:empty}

It should be noted that \prg{avmap} can produce maps with
zero nodes or one node as output.  Such maps are not actually
acceptible as input maps, since using them could lead to division
by zero in certain cases.  When \prg{avmap} produces such a map
it writes a warning message to the screen.

\chapter{The \prg{ross} program}\label{ch:ross}

The \prg{ross} program simply displays the Ross ordering of on $N$
elements.  The ordering was developed by \citeasnoun{Ross34}.  It is
an ordering of pairs of elements such as to maximize the distance
between pairs involving the same elements. 

The program, as an independent program, is not needed directly for
causal map analysis or elicitation.  The basic calculation is needed
for the \prg{askmap} program (chapter~\ref{ch:askmap}), and once that
was written, it took no more that a few minutes to write a stand
alone program.

It's usage is simple:
\begin{alltt}
 ross 7
\end{alltt}
will produce the Ross ordering for seven elements:
\begin{alltt}
1: 1 2
2: 7 3
3: 6 4
4: 5 1
5: 3 2
6: 4 7
7: 5 6
8: 1 3
9: 2 4
10: 7 5
11: 6 1
12: 4 3
13: 5 2
14: 6 7
15: 1 4
16: 3 5
17: 2 6
18: 7 1
19: 4 5
20: 3 6
21: 2 7
\end{alltt}

\section{Limitations}

\subsection{Limitations on $N$}

Running
\begin{alltt}
 ross 200
\end{alltt}
produces
\begin{alltt}
rossify(): must be no more than 181 elements
Error getting ross array
\end{alltt}

The 181 is not an arbitrary number (It is the greatest integer $N$
such that $(N^2-N)/2 \le 65536/4$.  On some computers the largest
single array you can request is 65536 bytes without writing machine
and compiler specific code, and 4 is the maximum size of the element
of such an array on such a machine.)  Due to the fact that
the essential routine used for generating the Ross ordering is actually
pretty stupid (and stores the whole array in memory) this limit exists.

Unless you have found a use of Ross ordering that is new to me, then
it is unlikely that you will need it for more than 30 elements.  For
the purposes of Ross ordering, if you really want more that 181
elements, you could safely give up on the Ross ordering, and just
order your pairs randomly.

\subsection{No \opt{-h} option}

The \prg{ross} program doesn't offer a usage message.


%%%%%%%%%%%%%%%%%%%%%%%%%%%%%%%%%%%%%%%%%%%%%%%%%%%%%%%%%%%%%%%%%%%%%%%%
%%%%%%%%%%%%%%%%%%%%%%%%%%%%%%%%%%%%%%%%%%%%%%%%%%%%%%%%%%%%%%%%%%%%%%%%
%%%%%         Appendices begin here, Do not put chapters after    %%%%%%
%%%%%           this point                                        %%%%%%
%%%%%%%%%%%%%%%%%%%%%%%%%%%%%%%%%%%%%%%%%%%%%%%%%%%%%%%%%%%%%%%%%%%%%%%%
%%%%%%%%%%%%%%%%%%%%%%%%%%%%%%%%%%%%%%%%%%%%%%%%%%%%%%%%%%%%%%%%%%%%%%%%

\appendix\renewcommand{\chaptername}{Appendix}
\chapter{GNU General Public License}\label{app:gnu}

\begin{center}
{\large GNU General Public License}\\
Version 2, June 1991\\[.6ex]
Copyright \copyright~1989, 1991 Free Software Foundation, Inc.\\
675 Mass Ave, Cambridge, MA 02139, USA\\[.2ex]
Everyone is permitted to copy and distribute verbatim copies
of this license document, but changing it is not allowed.
\end{center}

\section{Preamble}

The licenses for most software are designed to take away your
freedom to share and change it.  By contrast, the GNU General Public
License is intended to guarantee your freedom to share and change free
software\dash to make sure the software is free for all its users.  This
General Public License applies to most of the Free Software
Foundation's software and to any other program whose authors commit to
using it.  (Some other Free Software Foundation software is covered by
the GNU Library General Public License instead.)  You can apply it to
your programs, too.

  When we speak of free software, we are referring to freedom, not
price.  Our General Public Licenses are designed to make sure that you
have the freedom to distribute copies of free software (and charge for
this service if you wish), that you receive source code or can get it
if you want it, that you can change the software or use pieces of it
in new free programs; and that you know you can do these things.

  To protect your rights, we need to make restrictions that forbid
anyone to deny you these rights or to ask you to surrender the rights.
These restrictions translate to certain responsibilities for you if you
distribute copies of the software, or if you modify it.

  For example, if you distribute copies of such a program, whether
gratis or for a fee, you must give the recipients all the rights that
you have.  You must make sure that they, too, receive or can get the
source code.  And you must show them these terms so they know their
rights.

  We protect your rights with two steps: (1) copyright the software, and
(2) offer you this license which gives you legal permission to copy,
distribute and/or modify the software.

  Also, for each author's protection and ours, we want to make certain
that everyone understands that there is no warranty for this free
software.  If the software is modified by someone else and passed on, we
want its recipients to know that what they have is not the original, so
that any problems introduced by others will not reflect on the original
authors' reputations.

  Finally, any free program is threatened constantly by software
patents.  We wish to avoid the danger that redistributors of a free
program will individually obtain patent licenses, in effect making the
program proprietary.  To prevent this, we have made it clear that any
patent must be licensed for everyone's free use or not licensed at all.

  The precise terms and conditions for copying, distribution and
modification follow.

\section[Terms and Conditions]{GNU General Public License
   Terms and Conditions for Copying, Distribution and Modification}

\newcounter{gnucounter} \setcounter{gnucounter}{-1}
\newcommand{\theguncounter}{\arabic{gnucounter}}
\newcommand{\gnuitem}{%
  \par\vskip.4ex plus .2ex minus .2ex
  \noindent 	% no indent
  \stepcounter{gnucounter}%
  \textit{\thegnucounter.}\hskip1em\relax}

\gnuitem
This License applies to any program or other work which contains
a notice placed by the copyright holder saying it may be distributed
under the terms of this General Public License.  The ``Program'', below,
refers to any such program or work, and a ``work based on the Program''
means either the Program or any derivative work under copyright law:
that is to say, a work containing the Program or a portion of it,
either verbatim or with modifications and/or translated into another
language.  (Hereinafter, translation is included without limitation in
the term ``modification''.)  Each licensee is addressed as ``you''.

Activities other than copying, distribution and modification are not
covered by this License; they are outside its scope.  The act of
running the Program is not restricted, and the output from the Program
is covered only if its contents constitute a work based on the
Program (independent of having been made by running the Program).
Whether that is true depends on what the Program does.

\gnuitem 
You may copy and distribute verbatim copies of the Program's
source code as you receive it, in any medium, provided that you
conspicuously and appropriately publish on each copy an appropriate
copyright notice and disclaimer of warranty; keep intact all the
notices that refer to this License and to the absence of any warranty;
and give any other recipients of the Program a copy of this License
along with the Program.

You may charge a fee for the physical act of transferring a copy, and
you may at your option offer warranty protection in exchange for a fee.

\gnuitem 
You may modify your copy or copies of the Program or any portion
of it, thus forming a work based on the Program, and copy and
distribute such modifications or work under the terms of Section 1
above, provided that you also meet all of these conditions:
    \begin{itemize}
    \item[(a)] You must cause the modified files to carry prominent notices
    stating that you changed the files and the date of any change.

    \item[(b)] You must cause any work that you distribute or publish, that in
    whole or in part contains or is derived from the Program or any
    part thereof, to be licensed as a whole at no charge to all third
    parties under the terms of this License.

    \item[(c)] If the modified program normally reads commands interactively
    when run, you must cause it, when started running for such
    interactive use in the most ordinary way, to print or display an
    announcement including an appropriate copyright notice and a
    notice that there is no warranty (or else, saying that you provide
    a warranty) and that users may redistribute the program under
    these conditions, and telling the user how to view a copy of this
    License.  (Exception: if the Program itself is interactive but
    does not normally print such an announcement, your work based on
    the Program is not required to print an announcement.)
    \end{itemize}

These requirements apply to the modified work as a whole.  If
identifiable sections of that work are not derived from the Program,
and can be reasonably considered independent and separate works in
themselves, then this License, and its terms, do not apply to those
sections when you distribute them as separate works.  But when you
distribute the same sections as part of a whole which is a work based
on the Program, the distribution of the whole must be on the terms of
this License, whose permissions for other licensees extend to the
entire whole, and thus to each and every part regardless of who wrote it.

Thus, it is not the intent of this section to claim rights or contest
your rights to work written entirely by you; rather, the intent is to
exercise the right to control the distribution of derivative or
collective works based on the Program.

In addition, mere aggregation of another work not based on the Program
with the Program (or with a work based on the Program) on a volume of
a storage or distribution medium does not bring the other work under
the scope of this License.

\gnuitem % should be 3
You may copy and distribute the Program (or a work based on it,
under Section 2) in object code or executable form under the terms of
Sections 1 and 2 above provided that you also do one of the following:
    \begin{itemize} % should start at a

    \item[(a)] Accompany it with the complete corresponding machine-readable
    source code, which must be distributed under the terms of Sections
    1 and 2 above on a medium customarily used for software interchange; or,

    \item[(b)] Accompany it with a written offer, valid for at least three
    years, to give any third party, for a charge no more than your
    cost of physically performing source distribution, a complete
    machine-readable copy of the corresponding source code, to be
    distributed under the terms of Sections 1 and 2 above on a medium
    customarily used for software interchange; or,

    \item[(c)] Accompany it with the information you received as to the offer
    to distribute corresponding source code.  (This alternative is
    allowed only for noncommercial distribution and only if you
    received the program in object code or executable form with such
    an offer, in accord with Subsection b above.)
    \end{itemize}

The source code for a work means the preferred form of the work for
making modifications to it.  For an executable work, complete source
code means all the source code for all modules it contains, plus any
associated interface definition files, plus the scripts used to
control compilation and installation of the executable.  However, as a
special exception, the source code distributed need not include
anything that is normally distributed (in either source or binary
form) with the major components (compiler, kernel, and so on) of the
operating system on which the executable runs, unless that component
itself accompanies the executable.

If distribution of executable or object code is made by offering
access to copy from a designated place, then offering equivalent
access to copy the source code from the same place counts as
distribution of the source code, even though third parties are not
compelled to copy the source along with the object code.

\gnuitem % 4
You may not copy, modify, sublicense, or distribute the Program
except as expressly provided under this License.  Any attempt
otherwise to copy, modify, sublicense or distribute the Program is
void, and will automatically terminate your rights under this License.
However, parties who have received copies, or rights, from you under
this License will not have their licenses terminated so long as such
parties remain in full compliance.

\gnuitem %  5.
You are not required to accept this License, since you have not
signed it.  However, nothing else grants you permission to modify or
distribute the Program or its derivative works.  These actions are
prohibited by law if you do not accept this License.  Therefore, by
modifying or distributing the Program (or any work based on the
Program), you indicate your acceptance of this License to do so, and
all its terms and conditions for copying, distributing or modifying
the Program or works based on it.

\gnuitem % 6.
Each time you redistribute the Program (or any work based on the
Program), the recipient automatically receives a license from the
original licensor to copy, distribute or modify the Program subject to
these terms and conditions.  You may not impose any further
restrictions on the recipients' exercise of the rights granted herein.
You are not responsible for enforcing compliance by third parties to
this License.

\gnuitem % 7.
If, as a consequence of a court judgment or allegation of patent
infringement or for any other reason (not limited to patent issues),
conditions are imposed on you (whether by court order, agreement or
otherwise) that contradict the conditions of this License, they do not
excuse you from the conditions of this License.  If you cannot
distribute so as to satisfy simultaneously your obligations under this
License and any other pertinent obligations, then as a consequence you
may not distribute the Program at all.  For example, if a patent
license would not permit royalty-free redistribution of the Program by
all those who receive copies directly or indirectly through you, then
the only way you could satisfy both it and this License would be to
refrain entirely from distribution of the Program.

If any portion of this section is held invalid or unenforceable under
any particular circumstance, the balance of the section is intended to
apply and the section as a whole is intended to apply in other
circumstances.

It is not the purpose of this section to induce you to infringe any
patents or other property right claims or to contest validity of any
such claims; this section has the sole purpose of protecting the
integrity of the free software distribution system, which is
implemented by public license practices.  Many people have made
generous contributions to the wide range of software distributed
through that system in reliance on consistent application of that
system; it is up to the author/donor to decide if he or she is willing
to distribute software through any other system and a licensee cannot
impose that choice.

This section is intended to make thoroughly clear what is believed to
be a consequence of the rest of this License.

\gnuitem %  8.
If the distribution and/or use of the Program is restricted in
certain countries either by patents or by copyrighted interfaces, the
original copyright holder who places the Program under this License
may add an explicit geographical distribution limitation excluding
those countries, so that distribution is permitted only in or among
countries not thus excluded.  In such case, this License incorporates
the limitation as if written in the body of this License.

\gnuitem %  9.
The Free Software Foundation may publish revised and/or new versions
of the General Public License from time to time.  Such new versions will
be similar in spirit to the present version, but may differ in detail to
address new problems or concerns.

Each version is given a distinguishing version number.  If the Program
specifies a version number of this License which applies to it and ``any
later version'', you have the option of following the terms and conditions
either of that version or of any later version published by the Free
Software Foundation.  If the Program does not specify a version number of
this License, you may choose any version ever published by the Free Software
Foundation.

\gnuitem %  10.
If you wish to incorporate parts of the Program into other free
programs whose distribution conditions are different, write to the author
to ask for permission.  For software which is copyrighted by the Free
Software Foundation, write to the Free Software Foundation; we sometimes
make exceptions for this.  Our decision will be guided by the two goals
of preserving the free status of all derivatives of our free software and
of promoting the sharing and reuse of software generally.


\section{No Warranty}

\gnuitem %  11.
\begin{bfseries}
 Because the program is licensed free of charge, there is \emph{no warranty}
for the program, to the extent permitted by applicable law.  Except when
otherwise stated in writing the copyright holders and/or other parties
provide the program ``as is'' without warranty of any kind, either expressed
or implied, including, but not limited to, the implied warranties of
merchantability and fitness for a particular purpose.  The entire risk as
to the quality and performance of the program is with you.  Should the
program prove defective, you assume the cost of all necessary servicing,
repair or correction.
\end{bfseries}

\gnuitem % 12.
\begin{bfseries}
In no event unless required by applicable law or agreed to in writing
will any copyright holder, or any other party who may modify and/or
redistribute the program as permitted above, be liable to you for damages,
including any general, special, incidental or consequential damages arising
out of the use or inability to use the program (including but not limited
to loss of data or data being rendered inaccurate or losses sustained by
you or third parties or a failure of the program to operate with any other
programs), even if such holder or other party has been advised of the
possibility of such damages.
\end{bfseries}

\begin{center}\bfseries\large
End of Terms and Conditions
\end{center}

\section{How to apply these terms to your new programs}

If you develop a new program, and you want it to be of the greatest
possible use to the public, the best way to achieve this is to make it
free software which everyone can redistribute and change under these terms.

To do so, attach the following notices to the program.  It is safest
to attach them to the start of each source file to most effectively
convey the exclusion of warranty; and each file should have at least
the ``copyright'' line and a pointer to where the full notice is found.

\begin{quote}\ttfamily\raggedright
    <one line to give the program's name and a brief idea of what it does.>
    Copyright (C) 19yy  <name of author>

    This program is free software; you can redistribute it and/or modify
    it under the terms of the GNU General Public License as published by
    the Free Software Foundation; either version 2 of the License, or
    (at your option) any later version.

    This program is distributed in the hope that it will be useful,
    but WITHOUT ANY WARRANTY; without even the implied warranty of
    MERCHANTABILITY or FITNESS FOR A PARTICULAR PURPOSE.  See the
    GNU General Public License for more details.

    You should have received a copy of the GNU General Public License
    along with this program; if not, write to the Free Software
    Foundation, Inc., 675 Mass Ave, Cambridge, MA 02139, USA.
\end{quote}

Also add information on how to contact you by electronic and paper mail.

If the program is interactive, make it output a short notice like this
when it starts in an interactive mode:

\begin{quote}\ttfamily\raggedright
    Gnomovision version 69, Copyright (C) 19yy name of author
    Gnomovision comes with ABSOLUTELY NO WARRANTY; for details type `show w'.
    This is free software, and you are welcome to redistribute it
    under certain conditions; type `show c' for details.
\end{quote}

The hypothetical commands `show w' and `show c' should show the appropriate
parts of the General Public License.  Of course, the commands you use may
be called something other than `show w' and `show c'; they could even be
mouse-clicks or menu items\dash whatever suits your program.

You should also get your employer (if you work as a programmer) or your
school, if any, to sign a ``copyright disclaimer'' for the program, if
necessary.  Here is a sample; alter the names:

\begin{quote}\ttfamily\raggedright
  Yoyodyne, Inc., hereby disclaims all copyright interest in the program
  `Gnomovision' (which makes passes at compilers) written by James Hacker.

  <signature of Ty Coon>, 1 April 1989
  Ty Coon, President of Vice
\end{quote}

This General Public License does not permit incorporating your program into
proprietary programs.  If your program is a subroutine library, you may
consider it more useful to permit linking proprietary applications with the
library.  If this is what you want to do, use the GNU Library General
Public License instead of this License.

\chapter{Other software tools}\label{app:other}

This appendix, when it is written, will list contact information
for Mauri Laukkanen's CMAP2 package
\cite{Laukkanen90:ScandJM,Laukkanen92:Book},
and Colin Eden and associates Graphics Cope
\cite{AckermannETAL92:COPE,EdenAckermannCropper92:JMS}.
Some other information will listed to point
to influence diagramming software as well.


\bibliographystyle{jpgref}
\bibliography{drdoc,jom}
\end{document}
